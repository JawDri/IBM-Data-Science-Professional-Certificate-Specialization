\documentclass[11pt]{article}

    \usepackage[breakable]{tcolorbox}
    \usepackage{parskip} % Stop auto-indenting (to mimic markdown behaviour)
    
    \usepackage{iftex}
    \ifPDFTeX
    	\usepackage[T1]{fontenc}
    	\usepackage{mathpazo}
    \else
    	\usepackage{fontspec}
    \fi

    % Basic figure setup, for now with no caption control since it's done
    % automatically by Pandoc (which extracts ![](path) syntax from Markdown).
    \usepackage{graphicx}
    % Maintain compatibility with old templates. Remove in nbconvert 6.0
    \let\Oldincludegraphics\includegraphics
    % Ensure that by default, figures have no caption (until we provide a
    % proper Figure object with a Caption API and a way to capture that
    % in the conversion process - todo).
    \usepackage{caption}
    \DeclareCaptionFormat{nocaption}{}
    \captionsetup{format=nocaption,aboveskip=0pt,belowskip=0pt}

    \usepackage[Export]{adjustbox} % Used to constrain images to a maximum size
    \adjustboxset{max size={0.9\linewidth}{0.9\paperheight}}
    \usepackage{float}
    \floatplacement{figure}{H} % forces figures to be placed at the correct location
    \usepackage{xcolor} % Allow colors to be defined
    \usepackage{enumerate} % Needed for markdown enumerations to work
    \usepackage{geometry} % Used to adjust the document margins
    \usepackage{amsmath} % Equations
    \usepackage{amssymb} % Equations
    \usepackage{textcomp} % defines textquotesingle
    % Hack from http://tex.stackexchange.com/a/47451/13684:
    \AtBeginDocument{%
        \def\PYZsq{\textquotesingle}% Upright quotes in Pygmentized code
    }
    \usepackage{upquote} % Upright quotes for verbatim code
    \usepackage{eurosym} % defines \euro
    \usepackage[mathletters]{ucs} % Extended unicode (utf-8) support
    \usepackage{fancyvrb} % verbatim replacement that allows latex
    \usepackage{grffile} % extends the file name processing of package graphics 
                         % to support a larger range
    \makeatletter % fix for grffile with XeLaTeX
    \def\Gread@@xetex#1{%
      \IfFileExists{"\Gin@base".bb}%
      {\Gread@eps{\Gin@base.bb}}%
      {\Gread@@xetex@aux#1}%
    }
    \makeatother

    % The hyperref package gives us a pdf with properly built
    % internal navigation ('pdf bookmarks' for the table of contents,
    % internal cross-reference links, web links for URLs, etc.)
    \usepackage{hyperref}
    % The default LaTeX title has an obnoxious amount of whitespace. By default,
    % titling removes some of it. It also provides customization options.
    \usepackage{titling}
    \usepackage{longtable} % longtable support required by pandoc >1.10
    \usepackage{booktabs}  % table support for pandoc > 1.12.2
    \usepackage[inline]{enumitem} % IRkernel/repr support (it uses the enumerate* environment)
    \usepackage[normalem]{ulem} % ulem is needed to support strikethroughs (\sout)
                                % normalem makes italics be italics, not underlines
    \usepackage{mathrsfs}
    

    
    % Colors for the hyperref package
    \definecolor{urlcolor}{rgb}{0,.145,.698}
    \definecolor{linkcolor}{rgb}{.71,0.21,0.01}
    \definecolor{citecolor}{rgb}{.12,.54,.11}

    % ANSI colors
    \definecolor{ansi-black}{HTML}{3E424D}
    \definecolor{ansi-black-intense}{HTML}{282C36}
    \definecolor{ansi-red}{HTML}{E75C58}
    \definecolor{ansi-red-intense}{HTML}{B22B31}
    \definecolor{ansi-green}{HTML}{00A250}
    \definecolor{ansi-green-intense}{HTML}{007427}
    \definecolor{ansi-yellow}{HTML}{DDB62B}
    \definecolor{ansi-yellow-intense}{HTML}{B27D12}
    \definecolor{ansi-blue}{HTML}{208FFB}
    \definecolor{ansi-blue-intense}{HTML}{0065CA}
    \definecolor{ansi-magenta}{HTML}{D160C4}
    \definecolor{ansi-magenta-intense}{HTML}{A03196}
    \definecolor{ansi-cyan}{HTML}{60C6C8}
    \definecolor{ansi-cyan-intense}{HTML}{258F8F}
    \definecolor{ansi-white}{HTML}{C5C1B4}
    \definecolor{ansi-white-intense}{HTML}{A1A6B2}
    \definecolor{ansi-default-inverse-fg}{HTML}{FFFFFF}
    \definecolor{ansi-default-inverse-bg}{HTML}{000000}

    % commands and environments needed by pandoc snippets
    % extracted from the output of `pandoc -s`
    \providecommand{\tightlist}{%
      \setlength{\itemsep}{0pt}\setlength{\parskip}{0pt}}
    \DefineVerbatimEnvironment{Highlighting}{Verbatim}{commandchars=\\\{\}}
    % Add ',fontsize=\small' for more characters per line
    \newenvironment{Shaded}{}{}
    \newcommand{\KeywordTok}[1]{\textcolor[rgb]{0.00,0.44,0.13}{\textbf{{#1}}}}
    \newcommand{\DataTypeTok}[1]{\textcolor[rgb]{0.56,0.13,0.00}{{#1}}}
    \newcommand{\DecValTok}[1]{\textcolor[rgb]{0.25,0.63,0.44}{{#1}}}
    \newcommand{\BaseNTok}[1]{\textcolor[rgb]{0.25,0.63,0.44}{{#1}}}
    \newcommand{\FloatTok}[1]{\textcolor[rgb]{0.25,0.63,0.44}{{#1}}}
    \newcommand{\CharTok}[1]{\textcolor[rgb]{0.25,0.44,0.63}{{#1}}}
    \newcommand{\StringTok}[1]{\textcolor[rgb]{0.25,0.44,0.63}{{#1}}}
    \newcommand{\CommentTok}[1]{\textcolor[rgb]{0.38,0.63,0.69}{\textit{{#1}}}}
    \newcommand{\OtherTok}[1]{\textcolor[rgb]{0.00,0.44,0.13}{{#1}}}
    \newcommand{\AlertTok}[1]{\textcolor[rgb]{1.00,0.00,0.00}{\textbf{{#1}}}}
    \newcommand{\FunctionTok}[1]{\textcolor[rgb]{0.02,0.16,0.49}{{#1}}}
    \newcommand{\RegionMarkerTok}[1]{{#1}}
    \newcommand{\ErrorTok}[1]{\textcolor[rgb]{1.00,0.00,0.00}{\textbf{{#1}}}}
    \newcommand{\NormalTok}[1]{{#1}}
    
    % Additional commands for more recent versions of Pandoc
    \newcommand{\ConstantTok}[1]{\textcolor[rgb]{0.53,0.00,0.00}{{#1}}}
    \newcommand{\SpecialCharTok}[1]{\textcolor[rgb]{0.25,0.44,0.63}{{#1}}}
    \newcommand{\VerbatimStringTok}[1]{\textcolor[rgb]{0.25,0.44,0.63}{{#1}}}
    \newcommand{\SpecialStringTok}[1]{\textcolor[rgb]{0.73,0.40,0.53}{{#1}}}
    \newcommand{\ImportTok}[1]{{#1}}
    \newcommand{\DocumentationTok}[1]{\textcolor[rgb]{0.73,0.13,0.13}{\textit{{#1}}}}
    \newcommand{\AnnotationTok}[1]{\textcolor[rgb]{0.38,0.63,0.69}{\textbf{\textit{{#1}}}}}
    \newcommand{\CommentVarTok}[1]{\textcolor[rgb]{0.38,0.63,0.69}{\textbf{\textit{{#1}}}}}
    \newcommand{\VariableTok}[1]{\textcolor[rgb]{0.10,0.09,0.49}{{#1}}}
    \newcommand{\ControlFlowTok}[1]{\textcolor[rgb]{0.00,0.44,0.13}{\textbf{{#1}}}}
    \newcommand{\OperatorTok}[1]{\textcolor[rgb]{0.40,0.40,0.40}{{#1}}}
    \newcommand{\BuiltInTok}[1]{{#1}}
    \newcommand{\ExtensionTok}[1]{{#1}}
    \newcommand{\PreprocessorTok}[1]{\textcolor[rgb]{0.74,0.48,0.00}{{#1}}}
    \newcommand{\AttributeTok}[1]{\textcolor[rgb]{0.49,0.56,0.16}{{#1}}}
    \newcommand{\InformationTok}[1]{\textcolor[rgb]{0.38,0.63,0.69}{\textbf{\textit{{#1}}}}}
    \newcommand{\WarningTok}[1]{\textcolor[rgb]{0.38,0.63,0.69}{\textbf{\textit{{#1}}}}}
    
    
    % Define a nice break command that doesn't care if a line doesn't already
    % exist.
    \def\br{\hspace*{\fill} \\* }
    % Math Jax compatibility definitions
    \def\gt{>}
    \def\lt{<}
    \let\Oldtex\TeX
    \let\Oldlatex\LaTeX
    \renewcommand{\TeX}{\textrm{\Oldtex}}
    \renewcommand{\LaTeX}{\textrm{\Oldlatex}}
    % Document parameters
    % Document title
    \title{DV0101EN-3-5-1-Generating-Maps-in-Python-py-v2.0}
    
    
    
    
    
% Pygments definitions
\makeatletter
\def\PY@reset{\let\PY@it=\relax \let\PY@bf=\relax%
    \let\PY@ul=\relax \let\PY@tc=\relax%
    \let\PY@bc=\relax \let\PY@ff=\relax}
\def\PY@tok#1{\csname PY@tok@#1\endcsname}
\def\PY@toks#1+{\ifx\relax#1\empty\else%
    \PY@tok{#1}\expandafter\PY@toks\fi}
\def\PY@do#1{\PY@bc{\PY@tc{\PY@ul{%
    \PY@it{\PY@bf{\PY@ff{#1}}}}}}}
\def\PY#1#2{\PY@reset\PY@toks#1+\relax+\PY@do{#2}}

\expandafter\def\csname PY@tok@w\endcsname{\def\PY@tc##1{\textcolor[rgb]{0.73,0.73,0.73}{##1}}}
\expandafter\def\csname PY@tok@c\endcsname{\let\PY@it=\textit\def\PY@tc##1{\textcolor[rgb]{0.25,0.50,0.50}{##1}}}
\expandafter\def\csname PY@tok@cp\endcsname{\def\PY@tc##1{\textcolor[rgb]{0.74,0.48,0.00}{##1}}}
\expandafter\def\csname PY@tok@k\endcsname{\let\PY@bf=\textbf\def\PY@tc##1{\textcolor[rgb]{0.00,0.50,0.00}{##1}}}
\expandafter\def\csname PY@tok@kp\endcsname{\def\PY@tc##1{\textcolor[rgb]{0.00,0.50,0.00}{##1}}}
\expandafter\def\csname PY@tok@kt\endcsname{\def\PY@tc##1{\textcolor[rgb]{0.69,0.00,0.25}{##1}}}
\expandafter\def\csname PY@tok@o\endcsname{\def\PY@tc##1{\textcolor[rgb]{0.40,0.40,0.40}{##1}}}
\expandafter\def\csname PY@tok@ow\endcsname{\let\PY@bf=\textbf\def\PY@tc##1{\textcolor[rgb]{0.67,0.13,1.00}{##1}}}
\expandafter\def\csname PY@tok@nb\endcsname{\def\PY@tc##1{\textcolor[rgb]{0.00,0.50,0.00}{##1}}}
\expandafter\def\csname PY@tok@nf\endcsname{\def\PY@tc##1{\textcolor[rgb]{0.00,0.00,1.00}{##1}}}
\expandafter\def\csname PY@tok@nc\endcsname{\let\PY@bf=\textbf\def\PY@tc##1{\textcolor[rgb]{0.00,0.00,1.00}{##1}}}
\expandafter\def\csname PY@tok@nn\endcsname{\let\PY@bf=\textbf\def\PY@tc##1{\textcolor[rgb]{0.00,0.00,1.00}{##1}}}
\expandafter\def\csname PY@tok@ne\endcsname{\let\PY@bf=\textbf\def\PY@tc##1{\textcolor[rgb]{0.82,0.25,0.23}{##1}}}
\expandafter\def\csname PY@tok@nv\endcsname{\def\PY@tc##1{\textcolor[rgb]{0.10,0.09,0.49}{##1}}}
\expandafter\def\csname PY@tok@no\endcsname{\def\PY@tc##1{\textcolor[rgb]{0.53,0.00,0.00}{##1}}}
\expandafter\def\csname PY@tok@nl\endcsname{\def\PY@tc##1{\textcolor[rgb]{0.63,0.63,0.00}{##1}}}
\expandafter\def\csname PY@tok@ni\endcsname{\let\PY@bf=\textbf\def\PY@tc##1{\textcolor[rgb]{0.60,0.60,0.60}{##1}}}
\expandafter\def\csname PY@tok@na\endcsname{\def\PY@tc##1{\textcolor[rgb]{0.49,0.56,0.16}{##1}}}
\expandafter\def\csname PY@tok@nt\endcsname{\let\PY@bf=\textbf\def\PY@tc##1{\textcolor[rgb]{0.00,0.50,0.00}{##1}}}
\expandafter\def\csname PY@tok@nd\endcsname{\def\PY@tc##1{\textcolor[rgb]{0.67,0.13,1.00}{##1}}}
\expandafter\def\csname PY@tok@s\endcsname{\def\PY@tc##1{\textcolor[rgb]{0.73,0.13,0.13}{##1}}}
\expandafter\def\csname PY@tok@sd\endcsname{\let\PY@it=\textit\def\PY@tc##1{\textcolor[rgb]{0.73,0.13,0.13}{##1}}}
\expandafter\def\csname PY@tok@si\endcsname{\let\PY@bf=\textbf\def\PY@tc##1{\textcolor[rgb]{0.73,0.40,0.53}{##1}}}
\expandafter\def\csname PY@tok@se\endcsname{\let\PY@bf=\textbf\def\PY@tc##1{\textcolor[rgb]{0.73,0.40,0.13}{##1}}}
\expandafter\def\csname PY@tok@sr\endcsname{\def\PY@tc##1{\textcolor[rgb]{0.73,0.40,0.53}{##1}}}
\expandafter\def\csname PY@tok@ss\endcsname{\def\PY@tc##1{\textcolor[rgb]{0.10,0.09,0.49}{##1}}}
\expandafter\def\csname PY@tok@sx\endcsname{\def\PY@tc##1{\textcolor[rgb]{0.00,0.50,0.00}{##1}}}
\expandafter\def\csname PY@tok@m\endcsname{\def\PY@tc##1{\textcolor[rgb]{0.40,0.40,0.40}{##1}}}
\expandafter\def\csname PY@tok@gh\endcsname{\let\PY@bf=\textbf\def\PY@tc##1{\textcolor[rgb]{0.00,0.00,0.50}{##1}}}
\expandafter\def\csname PY@tok@gu\endcsname{\let\PY@bf=\textbf\def\PY@tc##1{\textcolor[rgb]{0.50,0.00,0.50}{##1}}}
\expandafter\def\csname PY@tok@gd\endcsname{\def\PY@tc##1{\textcolor[rgb]{0.63,0.00,0.00}{##1}}}
\expandafter\def\csname PY@tok@gi\endcsname{\def\PY@tc##1{\textcolor[rgb]{0.00,0.63,0.00}{##1}}}
\expandafter\def\csname PY@tok@gr\endcsname{\def\PY@tc##1{\textcolor[rgb]{1.00,0.00,0.00}{##1}}}
\expandafter\def\csname PY@tok@ge\endcsname{\let\PY@it=\textit}
\expandafter\def\csname PY@tok@gs\endcsname{\let\PY@bf=\textbf}
\expandafter\def\csname PY@tok@gp\endcsname{\let\PY@bf=\textbf\def\PY@tc##1{\textcolor[rgb]{0.00,0.00,0.50}{##1}}}
\expandafter\def\csname PY@tok@go\endcsname{\def\PY@tc##1{\textcolor[rgb]{0.53,0.53,0.53}{##1}}}
\expandafter\def\csname PY@tok@gt\endcsname{\def\PY@tc##1{\textcolor[rgb]{0.00,0.27,0.87}{##1}}}
\expandafter\def\csname PY@tok@err\endcsname{\def\PY@bc##1{\setlength{\fboxsep}{0pt}\fcolorbox[rgb]{1.00,0.00,0.00}{1,1,1}{\strut ##1}}}
\expandafter\def\csname PY@tok@kc\endcsname{\let\PY@bf=\textbf\def\PY@tc##1{\textcolor[rgb]{0.00,0.50,0.00}{##1}}}
\expandafter\def\csname PY@tok@kd\endcsname{\let\PY@bf=\textbf\def\PY@tc##1{\textcolor[rgb]{0.00,0.50,0.00}{##1}}}
\expandafter\def\csname PY@tok@kn\endcsname{\let\PY@bf=\textbf\def\PY@tc##1{\textcolor[rgb]{0.00,0.50,0.00}{##1}}}
\expandafter\def\csname PY@tok@kr\endcsname{\let\PY@bf=\textbf\def\PY@tc##1{\textcolor[rgb]{0.00,0.50,0.00}{##1}}}
\expandafter\def\csname PY@tok@bp\endcsname{\def\PY@tc##1{\textcolor[rgb]{0.00,0.50,0.00}{##1}}}
\expandafter\def\csname PY@tok@fm\endcsname{\def\PY@tc##1{\textcolor[rgb]{0.00,0.00,1.00}{##1}}}
\expandafter\def\csname PY@tok@vc\endcsname{\def\PY@tc##1{\textcolor[rgb]{0.10,0.09,0.49}{##1}}}
\expandafter\def\csname PY@tok@vg\endcsname{\def\PY@tc##1{\textcolor[rgb]{0.10,0.09,0.49}{##1}}}
\expandafter\def\csname PY@tok@vi\endcsname{\def\PY@tc##1{\textcolor[rgb]{0.10,0.09,0.49}{##1}}}
\expandafter\def\csname PY@tok@vm\endcsname{\def\PY@tc##1{\textcolor[rgb]{0.10,0.09,0.49}{##1}}}
\expandafter\def\csname PY@tok@sa\endcsname{\def\PY@tc##1{\textcolor[rgb]{0.73,0.13,0.13}{##1}}}
\expandafter\def\csname PY@tok@sb\endcsname{\def\PY@tc##1{\textcolor[rgb]{0.73,0.13,0.13}{##1}}}
\expandafter\def\csname PY@tok@sc\endcsname{\def\PY@tc##1{\textcolor[rgb]{0.73,0.13,0.13}{##1}}}
\expandafter\def\csname PY@tok@dl\endcsname{\def\PY@tc##1{\textcolor[rgb]{0.73,0.13,0.13}{##1}}}
\expandafter\def\csname PY@tok@s2\endcsname{\def\PY@tc##1{\textcolor[rgb]{0.73,0.13,0.13}{##1}}}
\expandafter\def\csname PY@tok@sh\endcsname{\def\PY@tc##1{\textcolor[rgb]{0.73,0.13,0.13}{##1}}}
\expandafter\def\csname PY@tok@s1\endcsname{\def\PY@tc##1{\textcolor[rgb]{0.73,0.13,0.13}{##1}}}
\expandafter\def\csname PY@tok@mb\endcsname{\def\PY@tc##1{\textcolor[rgb]{0.40,0.40,0.40}{##1}}}
\expandafter\def\csname PY@tok@mf\endcsname{\def\PY@tc##1{\textcolor[rgb]{0.40,0.40,0.40}{##1}}}
\expandafter\def\csname PY@tok@mh\endcsname{\def\PY@tc##1{\textcolor[rgb]{0.40,0.40,0.40}{##1}}}
\expandafter\def\csname PY@tok@mi\endcsname{\def\PY@tc##1{\textcolor[rgb]{0.40,0.40,0.40}{##1}}}
\expandafter\def\csname PY@tok@il\endcsname{\def\PY@tc##1{\textcolor[rgb]{0.40,0.40,0.40}{##1}}}
\expandafter\def\csname PY@tok@mo\endcsname{\def\PY@tc##1{\textcolor[rgb]{0.40,0.40,0.40}{##1}}}
\expandafter\def\csname PY@tok@ch\endcsname{\let\PY@it=\textit\def\PY@tc##1{\textcolor[rgb]{0.25,0.50,0.50}{##1}}}
\expandafter\def\csname PY@tok@cm\endcsname{\let\PY@it=\textit\def\PY@tc##1{\textcolor[rgb]{0.25,0.50,0.50}{##1}}}
\expandafter\def\csname PY@tok@cpf\endcsname{\let\PY@it=\textit\def\PY@tc##1{\textcolor[rgb]{0.25,0.50,0.50}{##1}}}
\expandafter\def\csname PY@tok@c1\endcsname{\let\PY@it=\textit\def\PY@tc##1{\textcolor[rgb]{0.25,0.50,0.50}{##1}}}
\expandafter\def\csname PY@tok@cs\endcsname{\let\PY@it=\textit\def\PY@tc##1{\textcolor[rgb]{0.25,0.50,0.50}{##1}}}

\def\PYZbs{\char`\\}
\def\PYZus{\char`\_}
\def\PYZob{\char`\{}
\def\PYZcb{\char`\}}
\def\PYZca{\char`\^}
\def\PYZam{\char`\&}
\def\PYZlt{\char`\<}
\def\PYZgt{\char`\>}
\def\PYZsh{\char`\#}
\def\PYZpc{\char`\%}
\def\PYZdl{\char`\$}
\def\PYZhy{\char`\-}
\def\PYZsq{\char`\'}
\def\PYZdq{\char`\"}
\def\PYZti{\char`\~}
% for compatibility with earlier versions
\def\PYZat{@}
\def\PYZlb{[}
\def\PYZrb{]}
\makeatother


    % For linebreaks inside Verbatim environment from package fancyvrb. 
    \makeatletter
        \newbox\Wrappedcontinuationbox 
        \newbox\Wrappedvisiblespacebox 
        \newcommand*\Wrappedvisiblespace {\textcolor{red}{\textvisiblespace}} 
        \newcommand*\Wrappedcontinuationsymbol {\textcolor{red}{\llap{\tiny$\m@th\hookrightarrow$}}} 
        \newcommand*\Wrappedcontinuationindent {3ex } 
        \newcommand*\Wrappedafterbreak {\kern\Wrappedcontinuationindent\copy\Wrappedcontinuationbox} 
        % Take advantage of the already applied Pygments mark-up to insert 
        % potential linebreaks for TeX processing. 
        %        {, <, #, %, $, ' and ": go to next line. 
        %        _, }, ^, &, >, - and ~: stay at end of broken line. 
        % Use of \textquotesingle for straight quote. 
        \newcommand*\Wrappedbreaksatspecials {% 
            \def\PYGZus{\discretionary{\char`\_}{\Wrappedafterbreak}{\char`\_}}% 
            \def\PYGZob{\discretionary{}{\Wrappedafterbreak\char`\{}{\char`\{}}% 
            \def\PYGZcb{\discretionary{\char`\}}{\Wrappedafterbreak}{\char`\}}}% 
            \def\PYGZca{\discretionary{\char`\^}{\Wrappedafterbreak}{\char`\^}}% 
            \def\PYGZam{\discretionary{\char`\&}{\Wrappedafterbreak}{\char`\&}}% 
            \def\PYGZlt{\discretionary{}{\Wrappedafterbreak\char`\<}{\char`\<}}% 
            \def\PYGZgt{\discretionary{\char`\>}{\Wrappedafterbreak}{\char`\>}}% 
            \def\PYGZsh{\discretionary{}{\Wrappedafterbreak\char`\#}{\char`\#}}% 
            \def\PYGZpc{\discretionary{}{\Wrappedafterbreak\char`\%}{\char`\%}}% 
            \def\PYGZdl{\discretionary{}{\Wrappedafterbreak\char`\$}{\char`\$}}% 
            \def\PYGZhy{\discretionary{\char`\-}{\Wrappedafterbreak}{\char`\-}}% 
            \def\PYGZsq{\discretionary{}{\Wrappedafterbreak\textquotesingle}{\textquotesingle}}% 
            \def\PYGZdq{\discretionary{}{\Wrappedafterbreak\char`\"}{\char`\"}}% 
            \def\PYGZti{\discretionary{\char`\~}{\Wrappedafterbreak}{\char`\~}}% 
        } 
        % Some characters . , ; ? ! / are not pygmentized. 
        % This macro makes them "active" and they will insert potential linebreaks 
        \newcommand*\Wrappedbreaksatpunct {% 
            \lccode`\~`\.\lowercase{\def~}{\discretionary{\hbox{\char`\.}}{\Wrappedafterbreak}{\hbox{\char`\.}}}% 
            \lccode`\~`\,\lowercase{\def~}{\discretionary{\hbox{\char`\,}}{\Wrappedafterbreak}{\hbox{\char`\,}}}% 
            \lccode`\~`\;\lowercase{\def~}{\discretionary{\hbox{\char`\;}}{\Wrappedafterbreak}{\hbox{\char`\;}}}% 
            \lccode`\~`\:\lowercase{\def~}{\discretionary{\hbox{\char`\:}}{\Wrappedafterbreak}{\hbox{\char`\:}}}% 
            \lccode`\~`\?\lowercase{\def~}{\discretionary{\hbox{\char`\?}}{\Wrappedafterbreak}{\hbox{\char`\?}}}% 
            \lccode`\~`\!\lowercase{\def~}{\discretionary{\hbox{\char`\!}}{\Wrappedafterbreak}{\hbox{\char`\!}}}% 
            \lccode`\~`\/\lowercase{\def~}{\discretionary{\hbox{\char`\/}}{\Wrappedafterbreak}{\hbox{\char`\/}}}% 
            \catcode`\.\active
            \catcode`\,\active 
            \catcode`\;\active
            \catcode`\:\active
            \catcode`\?\active
            \catcode`\!\active
            \catcode`\/\active 
            \lccode`\~`\~ 	
        }
    \makeatother

    \let\OriginalVerbatim=\Verbatim
    \makeatletter
    \renewcommand{\Verbatim}[1][1]{%
        %\parskip\z@skip
        \sbox\Wrappedcontinuationbox {\Wrappedcontinuationsymbol}%
        \sbox\Wrappedvisiblespacebox {\FV@SetupFont\Wrappedvisiblespace}%
        \def\FancyVerbFormatLine ##1{\hsize\linewidth
            \vtop{\raggedright\hyphenpenalty\z@\exhyphenpenalty\z@
                \doublehyphendemerits\z@\finalhyphendemerits\z@
                \strut ##1\strut}%
        }%
        % If the linebreak is at a space, the latter will be displayed as visible
        % space at end of first line, and a continuation symbol starts next line.
        % Stretch/shrink are however usually zero for typewriter font.
        \def\FV@Space {%
            \nobreak\hskip\z@ plus\fontdimen3\font minus\fontdimen4\font
            \discretionary{\copy\Wrappedvisiblespacebox}{\Wrappedafterbreak}
            {\kern\fontdimen2\font}%
        }%
        
        % Allow breaks at special characters using \PYG... macros.
        \Wrappedbreaksatspecials
        % Breaks at punctuation characters . , ; ? ! and / need catcode=\active 	
        \OriginalVerbatim[#1,codes*=\Wrappedbreaksatpunct]%
    }
    \makeatother

    % Exact colors from NB
    \definecolor{incolor}{HTML}{303F9F}
    \definecolor{outcolor}{HTML}{D84315}
    \definecolor{cellborder}{HTML}{CFCFCF}
    \definecolor{cellbackground}{HTML}{F7F7F7}
    
    % prompt
    \makeatletter
    \newcommand{\boxspacing}{\kern\kvtcb@left@rule\kern\kvtcb@boxsep}
    \makeatother
    \newcommand{\prompt}[4]{
        \ttfamily\llap{{\color{#2}[#3]:\hspace{3pt}#4}}\vspace{-\baselineskip}
    }
    

    
    % Prevent overflowing lines due to hard-to-break entities
    \sloppy 
    % Setup hyperref package
    \hypersetup{
      breaklinks=true,  % so long urls are correctly broken across lines
      colorlinks=true,
      urlcolor=urlcolor,
      linkcolor=linkcolor,
      citecolor=citecolor,
      }
    % Slightly bigger margins than the latex defaults
    
    \geometry{verbose,tmargin=1in,bmargin=1in,lmargin=1in,rmargin=1in}
    
    

\begin{document}
    
    \maketitle
    
    

    
    Generating Maps with Python

    \hypertarget{introduction}{%
\subsection{Introduction}\label{introduction}}

In this lab, we will learn how to create maps for different objectives.
To do that, we will part ways with Matplotlib and work with another
Python visualization library, namely \textbf{Folium}. What is nice about
\textbf{Folium} is that it was developed for the sole purpose of
visualizing geospatial data. While other libraries are available to
visualize geospatial data, such as \textbf{plotly}, they might have a
cap on how many API calls you can make within a defined time frame.
\textbf{Folium}, on the other hand, is completely free.

    \hypertarget{table-of-contents}{%
\subsection{Table of Contents}\label{table-of-contents}}

\begin{enumerate}
\def\labelenumi{\arabic{enumi}.}
\tightlist
\item
  Section \ref{0}
\item
  Section \ref{2}
\item
  Section \ref{4}
\item
  Section \ref{6}
\item
  Section \ref{8}
\end{enumerate}

    \hypertarget{exploring-datasets-with-pandas-and-matplotlib}{%
\section{\texorpdfstring{Exploring Datasets with \emph{pandas} and
Matplotlib}{Exploring Datasets with pandas and Matplotlib}}\label{exploring-datasets-with-pandas-and-matplotlib}}

Toolkits: This lab heavily relies on
\href{http://pandas.pydata.org/}{\emph{pandas}} and
\href{http://www.numpy.org/}{\textbf{Numpy}} for data wrangling,
analysis, and visualization. The primary plotting library we will
explore in this lab is
\href{https://github.com/python-visualization/folium/}{\textbf{Folium}}.

Datasets:

\begin{enumerate}
\def\labelenumi{\arabic{enumi}.}
\item
  San Francisco Police Department Incidents for the year 2016 -
  \href{https://data.sfgov.org/Public-Safety/Police-Department-Incidents-Previous-Year-2016-/ritf-b9ki}{Police
  Department Incidents} from San Francisco public data portal. Incidents
  derived from San Francisco Police Department (SFPD) Crime Incident
  Reporting system. Updated daily, showing data for the entire year of
  2016. Address and location has been anonymized by moving to mid-block
  or to an intersection.
\item
  Immigration to Canada from 1980 to 2013 -
  \href{http://www.un.org/en/development/desa/population/migration/data/empirical2/migrationflows.shtml}{International
  migration flows to and from selected countries - The 2015 revision}
  from United Nation's website. The dataset contains annual data on the
  flows of international migrants as recorded by the countries of
  destination. The data presents both inflows and outflows according to
  the place of birth, citizenship or place of previous / next residence
  both for foreigners and nationals. For this lesson, we will focus on
  the Canadian Immigration data
\end{enumerate}

    \hypertarget{downloading-and-prepping-data}{%
\section{\texorpdfstring{Downloading and Prepping Data
}{Downloading and Prepping Data }}\label{downloading-and-prepping-data}}

    Import Primary Modules:

    \begin{tcolorbox}[breakable, size=fbox, boxrule=1pt, pad at break*=1mm,colback=cellbackground, colframe=cellborder]
\prompt{In}{incolor}{1}{\boxspacing}
\begin{Verbatim}[commandchars=\\\{\}]
\PY{k+kn}{import} \PY{n+nn}{numpy} \PY{k}{as} \PY{n+nn}{np}  \PY{c+c1}{\PYZsh{} useful for many scientific computing in Python}
\PY{k+kn}{import} \PY{n+nn}{pandas} \PY{k}{as} \PY{n+nn}{pd} \PY{c+c1}{\PYZsh{} primary data structure library}
\end{Verbatim}
\end{tcolorbox}

    \hypertarget{introduction-to-folium}{%
\section{\texorpdfstring{Introduction to Folium
}{Introduction to Folium }}\label{introduction-to-folium}}

    Folium is a powerful Python library that helps you create several types
of Leaflet maps. The fact that the Folium results are interactive makes
this library very useful for dashboard building.

From the official Folium documentation page:

\begin{quote}
Folium builds on the data wrangling strengths of the Python ecosystem
and the mapping strengths of the Leaflet.js library. Manipulate your
data in Python, then visualize it in on a Leaflet map via Folium.
\end{quote}

\begin{quote}
Folium makes it easy to visualize data that's been manipulated in Python
on an interactive Leaflet map. It enables both the binding of data to a
map for choropleth visualizations as well as passing Vincent/Vega
visualizations as markers on the map.
\end{quote}

\begin{quote}
The library has a number of built-in tilesets from OpenStreetMap,
Mapbox, and Stamen, and supports custom tilesets with Mapbox or
Cloudmade API keys. Folium supports both GeoJSON and TopoJSON overlays,
as well as the binding of data to those overlays to create choropleth
maps with color-brewer color schemes.
\end{quote}

    \hypertarget{lets-install-folium}{%
\paragraph{\texorpdfstring{Let's install
\textbf{Folium}}{Let's install Folium}}\label{lets-install-folium}}

    \textbf{Folium} is not available by default. So, we first need to
install it before we are able to import it.

    \begin{tcolorbox}[breakable, size=fbox, boxrule=1pt, pad at break*=1mm,colback=cellbackground, colframe=cellborder]
\prompt{In}{incolor}{2}{\boxspacing}
\begin{Verbatim}[commandchars=\\\{\}]
\PY{o}{!}conda install \PYZhy{}c conda\PYZhy{}forge \PY{n+nv}{folium}\PY{o}{=}\PY{l+m}{0}.5.0 \PYZhy{}\PYZhy{}yes
\PY{k+kn}{import} \PY{n+nn}{folium}

\PY{n+nb}{print}\PY{p}{(}\PY{l+s+s1}{\PYZsq{}}\PY{l+s+s1}{Folium installed and imported!}\PY{l+s+s1}{\PYZsq{}}\PY{p}{)}
\end{Verbatim}
\end{tcolorbox}

    \begin{Verbatim}[commandchars=\\\{\}]
Solving environment: done


==> WARNING: A newer version of conda exists. <==
  current version: 4.5.11
  latest version: 4.8.0

Please update conda by running

    \$ conda update -n base -c defaults conda



\#\# Package Plan \#\#

  environment location: /home/jupyterlab/conda/envs/python

  added / updated specs:
    - folium=0.5.0


The following packages will be downloaded:

    package                    |            build
    ---------------------------|-----------------
    scikit-learn-0.20.1        |   py36h22eb022\_0         5.7 MB
    certifi-2019.11.28         |           py36\_0         149 KB  conda-forge
    liblapack-3.8.0            |      11\_openblas          10 KB  conda-forge
    liblapacke-3.8.0           |      11\_openblas          10 KB  conda-forge
    libopenblas-0.3.6          |       h5a2b251\_2         7.7 MB
    numpy-1.17.3               |   py36h95a1406\_0         5.2 MB  conda-forge
    scipy-1.4.1                |   py36h921218d\_0        18.9 MB  conda-forge
    libcblas-3.8.0             |      11\_openblas          10 KB  conda-forge
    libblas-3.8.0              |      11\_openblas          10 KB  conda-forge
    blas-2.11                  |         openblas          10 KB  conda-forge
    ------------------------------------------------------------
                                           Total:        37.7 MB

The following NEW packages will be INSTALLED:

    libblas:      3.8.0-11\_openblas                      conda-forge
    libcblas:     3.8.0-11\_openblas                      conda-forge
    liblapack:    3.8.0-11\_openblas                      conda-forge
    liblapacke:   3.8.0-11\_openblas                      conda-forge
    libopenblas:  0.3.6-h5a2b251\_2

The following packages will be UPDATED:

    blas:         1.1-openblas                           conda-forge -->
2.11-openblas         conda-forge
    certifi:      2019.9.11-py36\_0                       conda-forge -->
2019.11.28-py36\_0     conda-forge
    numpy:        1.16.2-py36\_blas\_openblash1522bff\_0    conda-forge
[blas\_openblas] --> 1.17.3-py36h95a1406\_0 conda-forge
    scipy:        1.2.1-py36\_blas\_openblash1522bff\_0     conda-forge
[blas\_openblas] --> 1.4.1-py36h921218d\_0  conda-forge

The following packages will be DOWNGRADED:

    scikit-learn: 0.20.1-py36\_blas\_openblashebff5e3\_1200 conda-forge
[blas\_openblas] --> 0.20.1-py36h22eb022\_0


Downloading and Extracting Packages
scikit-learn-0.20.1  | 5.7 MB    | \#\#\#\#\#\#\#\#\#\#\#\#\#\#\#\#\#\#\#\#\#\#\#\#\#\#\#\#\#\#\#\#\#\#\#\#\# | 100\%
certifi-2019.11.28   | 149 KB    | \#\#\#\#\#\#\#\#\#\#\#\#\#\#\#\#\#\#\#\#\#\#\#\#\#\#\#\#\#\#\#\#\#\#\#\#\# | 100\%
liblapack-3.8.0      | 10 KB     | \#\#\#\#\#\#\#\#\#\#\#\#\#\#\#\#\#\#\#\#\#\#\#\#\#\#\#\#\#\#\#\#\#\#\#\#\# | 100\%
liblapacke-3.8.0     | 10 KB     | \#\#\#\#\#\#\#\#\#\#\#\#\#\#\#\#\#\#\#\#\#\#\#\#\#\#\#\#\#\#\#\#\#\#\#\#\# | 100\%
libopenblas-0.3.6    | 7.7 MB    | \#\#\#\#\#\#\#\#\#\#\#\#\#\#\#\#\#\#\#\#\#\#\#\#\#\#\#\#\#\#\#\#\#\#\#\#\# | 100\%
numpy-1.17.3         | 5.2 MB    | \#\#\#\#\#\#\#\#\#\#\#\#\#\#\#\#\#\#\#\#\#\#\#\#\#\#\#\#\#\#\#\#\#\#\#\#\# | 100\%
scipy-1.4.1          | 18.9 MB   | \#\#\#\#\#\#\#\#\#\#\#\#\#\#\#\#\#\#\#\#\#\#\#\#\#\#\#\#\#\#\#\#\#\#\#\#\# | 100\%
libcblas-3.8.0       | 10 KB     | \#\#\#\#\#\#\#\#\#\#\#\#\#\#\#\#\#\#\#\#\#\#\#\#\#\#\#\#\#\#\#\#\#\#\#\#\# | 100\%
libblas-3.8.0        | 10 KB     | \#\#\#\#\#\#\#\#\#\#\#\#\#\#\#\#\#\#\#\#\#\#\#\#\#\#\#\#\#\#\#\#\#\#\#\#\# | 100\%
blas-2.11            | 10 KB     | \#\#\#\#\#\#\#\#\#\#\#\#\#\#\#\#\#\#\#\#\#\#\#\#\#\#\#\#\#\#\#\#\#\#\#\#\# | 100\%
Preparing transaction: done
Verifying transaction: done
Executing transaction: done
Folium installed and imported!
    \end{Verbatim}

    Generating the world map is straigtforward in \textbf{Folium}. You
simply create a \textbf{Folium} \emph{Map} object and then you display
it. What is attactive about \textbf{Folium} maps is that they are
interactive, so you can zoom into any region of interest despite the
initial zoom level.

    \begin{tcolorbox}[breakable, size=fbox, boxrule=1pt, pad at break*=1mm,colback=cellbackground, colframe=cellborder]
\prompt{In}{incolor}{3}{\boxspacing}
\begin{Verbatim}[commandchars=\\\{\}]
\PY{c+c1}{\PYZsh{} define the world map}
\PY{n}{world\PYZus{}map} \PY{o}{=} \PY{n}{folium}\PY{o}{.}\PY{n}{Map}\PY{p}{(}\PY{p}{)}

\PY{c+c1}{\PYZsh{} display world map}
\PY{n}{world\PYZus{}map}
\end{Verbatim}
\end{tcolorbox}

            \begin{tcolorbox}[breakable, size=fbox, boxrule=.5pt, pad at break*=1mm, opacityfill=0]
\prompt{Out}{outcolor}{3}{\boxspacing}
\begin{Verbatim}[commandchars=\\\{\}]
<folium.folium.Map at 0x7faf38030eb8>
\end{Verbatim}
\end{tcolorbox}
        
    Go ahead. Try zooming in and out of the rendered map above.

    You can customize this default definition of the world map by specifying
the centre of your map and the intial zoom level.

All locations on a map are defined by their respective \emph{Latitude}
and \emph{Longitude} values. So you can create a map and pass in a
center of \emph{Latitude} and \emph{Longitude} values of \textbf{{[}0,
0{]}}.

For a defined center, you can also define the intial zoom level into
that location when the map is rendered. \textbf{The higher the zoom
level the more the map is zoomed into the center}.

Let's create a map centered around Canada and play with the zoom level
to see how it affects the rendered map.

    \begin{tcolorbox}[breakable, size=fbox, boxrule=1pt, pad at break*=1mm,colback=cellbackground, colframe=cellborder]
\prompt{In}{incolor}{4}{\boxspacing}
\begin{Verbatim}[commandchars=\\\{\}]
\PY{c+c1}{\PYZsh{} define the world map centered around Canada with a low zoom level}
\PY{n}{world\PYZus{}map} \PY{o}{=} \PY{n}{folium}\PY{o}{.}\PY{n}{Map}\PY{p}{(}\PY{n}{location}\PY{o}{=}\PY{p}{[}\PY{l+m+mf}{56.130}\PY{p}{,} \PY{o}{\PYZhy{}}\PY{l+m+mf}{106.35}\PY{p}{]}\PY{p}{,} \PY{n}{zoom\PYZus{}start}\PY{o}{=}\PY{l+m+mi}{4}\PY{p}{)}

\PY{c+c1}{\PYZsh{} display world map}
\PY{n}{world\PYZus{}map}
\end{Verbatim}
\end{tcolorbox}

            \begin{tcolorbox}[breakable, size=fbox, boxrule=.5pt, pad at break*=1mm, opacityfill=0]
\prompt{Out}{outcolor}{4}{\boxspacing}
\begin{Verbatim}[commandchars=\\\{\}]
<folium.folium.Map at 0x7faf1a0ab4e0>
\end{Verbatim}
\end{tcolorbox}
        
    Let's create the map again with a higher zoom level

    \begin{tcolorbox}[breakable, size=fbox, boxrule=1pt, pad at break*=1mm,colback=cellbackground, colframe=cellborder]
\prompt{In}{incolor}{5}{\boxspacing}
\begin{Verbatim}[commandchars=\\\{\}]
\PY{c+c1}{\PYZsh{} define the world map centered around Canada with a higher zoom level}
\PY{n}{world\PYZus{}map} \PY{o}{=} \PY{n}{folium}\PY{o}{.}\PY{n}{Map}\PY{p}{(}\PY{n}{location}\PY{o}{=}\PY{p}{[}\PY{l+m+mf}{56.130}\PY{p}{,} \PY{o}{\PYZhy{}}\PY{l+m+mf}{106.35}\PY{p}{]}\PY{p}{,} \PY{n}{zoom\PYZus{}start}\PY{o}{=}\PY{l+m+mi}{8}\PY{p}{)}

\PY{c+c1}{\PYZsh{} display world map}
\PY{n}{world\PYZus{}map}
\end{Verbatim}
\end{tcolorbox}

            \begin{tcolorbox}[breakable, size=fbox, boxrule=.5pt, pad at break*=1mm, opacityfill=0]
\prompt{Out}{outcolor}{5}{\boxspacing}
\begin{Verbatim}[commandchars=\\\{\}]
<folium.folium.Map at 0x7faf1a03b438>
\end{Verbatim}
\end{tcolorbox}
        
    As you can see, the higher the zoom level the more the map is zoomed
into the given center.

    \textbf{Question}: Create a map of Mexico with a zoom level of 4.

    \begin{tcolorbox}[breakable, size=fbox, boxrule=1pt, pad at break*=1mm,colback=cellbackground, colframe=cellborder]
\prompt{In}{incolor}{6}{\boxspacing}
\begin{Verbatim}[commandchars=\\\{\}]
\PY{c+c1}{\PYZsh{}\PYZsh{}\PYZsh{} type your answer here}
\PY{n}{mexico\PYZus{}latitude} \PY{o}{=} \PY{l+m+mf}{23.6345} 
\PY{n}{mexico\PYZus{}longitude} \PY{o}{=} \PY{o}{\PYZhy{}}\PY{l+m+mf}{102.5528}
\PY{n}{mexico\PYZus{}map} \PY{o}{=} \PY{n}{folium}\PY{o}{.}\PY{n}{Map}\PY{p}{(}\PY{n}{location}\PY{o}{=}\PY{p}{[}\PY{n}{mexico\PYZus{}latitude}\PY{p}{,} \PY{n}{mexico\PYZus{}longitude}\PY{p}{]}\PY{p}{,} \PY{n}{zoom\PYZus{}start}\PY{o}{=}\PY{l+m+mi}{4}\PY{p}{)}
\PY{n}{mexico\PYZus{}map}
\end{Verbatim}
\end{tcolorbox}

            \begin{tcolorbox}[breakable, size=fbox, boxrule=.5pt, pad at break*=1mm, opacityfill=0]
\prompt{Out}{outcolor}{6}{\boxspacing}
\begin{Verbatim}[commandchars=\\\{\}]
<folium.folium.Map at 0x7faf1a040438>
\end{Verbatim}
\end{tcolorbox}
        
    Double-click \textbf{here} for the solution.

    Another cool feature of \textbf{Folium} is that you can generate
different map styles.

    \hypertarget{a.-stamen-toner-maps}{%
\subsubsection{A. Stamen Toner Maps}\label{a.-stamen-toner-maps}}

These are high-contrast B+W (black and white) maps. They are perfect for
data mashups and exploring river meanders and coastal zones.

    Let's create a Stamen Toner map of canada with a zoom level of 4.

    \begin{tcolorbox}[breakable, size=fbox, boxrule=1pt, pad at break*=1mm,colback=cellbackground, colframe=cellborder]
\prompt{In}{incolor}{7}{\boxspacing}
\begin{Verbatim}[commandchars=\\\{\}]
\PY{c+c1}{\PYZsh{} create a Stamen Toner map of the world centered around Canada}
\PY{n}{world\PYZus{}map} \PY{o}{=} \PY{n}{folium}\PY{o}{.}\PY{n}{Map}\PY{p}{(}\PY{n}{location}\PY{o}{=}\PY{p}{[}\PY{l+m+mf}{56.130}\PY{p}{,} \PY{o}{\PYZhy{}}\PY{l+m+mf}{106.35}\PY{p}{]}\PY{p}{,} \PY{n}{zoom\PYZus{}start}\PY{o}{=}\PY{l+m+mi}{4}\PY{p}{,} \PY{n}{tiles}\PY{o}{=}\PY{l+s+s1}{\PYZsq{}}\PY{l+s+s1}{Stamen Toner}\PY{l+s+s1}{\PYZsq{}}\PY{p}{)}

\PY{c+c1}{\PYZsh{} display map}
\PY{n}{world\PYZus{}map}
\end{Verbatim}
\end{tcolorbox}

            \begin{tcolorbox}[breakable, size=fbox, boxrule=.5pt, pad at break*=1mm, opacityfill=0]
\prompt{Out}{outcolor}{7}{\boxspacing}
\begin{Verbatim}[commandchars=\\\{\}]
<folium.folium.Map at 0x7faf1a040390>
\end{Verbatim}
\end{tcolorbox}
        
    Feel free to zoom in and out to see how this style compares to the
default one.

    \hypertarget{b.-stamen-terrain-maps}{%
\subsubsection{B. Stamen Terrain Maps}\label{b.-stamen-terrain-maps}}

These are maps that feature hill shading and natural vegetation colors.
They showcase advanced labeling and linework generalization of
dual-carriageway roads.

    Let's create a Stamen Terrain map of Canada with zoom level 4.

    \begin{tcolorbox}[breakable, size=fbox, boxrule=1pt, pad at break*=1mm,colback=cellbackground, colframe=cellborder]
\prompt{In}{incolor}{8}{\boxspacing}
\begin{Verbatim}[commandchars=\\\{\}]
\PY{c+c1}{\PYZsh{} create a Stamen Toner map of the world centered around Canada}
\PY{n}{world\PYZus{}map} \PY{o}{=} \PY{n}{folium}\PY{o}{.}\PY{n}{Map}\PY{p}{(}\PY{n}{location}\PY{o}{=}\PY{p}{[}\PY{l+m+mf}{56.130}\PY{p}{,} \PY{o}{\PYZhy{}}\PY{l+m+mf}{106.35}\PY{p}{]}\PY{p}{,} \PY{n}{zoom\PYZus{}start}\PY{o}{=}\PY{l+m+mi}{4}\PY{p}{,} \PY{n}{tiles}\PY{o}{=}\PY{l+s+s1}{\PYZsq{}}\PY{l+s+s1}{Stamen Terrain}\PY{l+s+s1}{\PYZsq{}}\PY{p}{)}

\PY{c+c1}{\PYZsh{} display map}
\PY{n}{world\PYZus{}map}
\end{Verbatim}
\end{tcolorbox}

            \begin{tcolorbox}[breakable, size=fbox, boxrule=.5pt, pad at break*=1mm, opacityfill=0]
\prompt{Out}{outcolor}{8}{\boxspacing}
\begin{Verbatim}[commandchars=\\\{\}]
<folium.folium.Map at 0x7faf1a065128>
\end{Verbatim}
\end{tcolorbox}
        
    Feel free to zoom in and out to see how this style compares to Stamen
Toner and the default style.

    \hypertarget{c.-mapbox-bright-maps}{%
\subsubsection{C. Mapbox Bright Maps}\label{c.-mapbox-bright-maps}}

These are maps that quite similar to the default style, except that the
borders are not visible with a low zoom level. Furthermore, unlike the
default style where country names are displayed in each country's native
language, \emph{Mapbox Bright} style displays all country names in
English.

    Let's create a world map with this style.

    \begin{tcolorbox}[breakable, size=fbox, boxrule=1pt, pad at break*=1mm,colback=cellbackground, colframe=cellborder]
\prompt{In}{incolor}{10}{\boxspacing}
\begin{Verbatim}[commandchars=\\\{\}]
\PY{c+c1}{\PYZsh{} create a world map with a Mapbox Bright style.}
\PY{n}{world\PYZus{}map} \PY{o}{=} \PY{n}{folium}\PY{o}{.}\PY{n}{Map}\PY{p}{(}\PY{n}{tiles}\PY{o}{=}\PY{l+s+s1}{\PYZsq{}}\PY{l+s+s1}{Mapbox Bright}\PY{l+s+s1}{\PYZsq{}}\PY{p}{)}

\PY{c+c1}{\PYZsh{} display the map}
\PY{n}{world\PYZus{}map}
\end{Verbatim}
\end{tcolorbox}

            \begin{tcolorbox}[breakable, size=fbox, boxrule=.5pt, pad at break*=1mm, opacityfill=0]
\prompt{Out}{outcolor}{10}{\boxspacing}
\begin{Verbatim}[commandchars=\\\{\}]
<folium.folium.Map at 0x7faf1a003c50>
\end{Verbatim}
\end{tcolorbox}
        
    Zoom in and notice how the borders start showing as you zoom in, and the
displayed country names are in English.

    \textbf{Question}: Create a map of Mexico to visualize its hill shading
and natural vegetation. Use a zoom level of 6.

    \begin{tcolorbox}[breakable, size=fbox, boxrule=1pt, pad at break*=1mm,colback=cellbackground, colframe=cellborder]
\prompt{In}{incolor}{11}{\boxspacing}
\begin{Verbatim}[commandchars=\\\{\}]
\PY{c+c1}{\PYZsh{}\PYZsh{}\PYZsh{} type your answer here}

\PY{n}{mexico\PYZus{}map} \PY{o}{=} \PY{n}{folium}\PY{o}{.}\PY{n}{Map}\PY{p}{(}\PY{n}{location}\PY{o}{=}\PY{p}{[}\PY{n}{mexico\PYZus{}latitude}\PY{p}{,} \PY{n}{mexico\PYZus{}longitude}\PY{p}{]}\PY{p}{,} \PY{n}{zoom\PYZus{}start}\PY{o}{=}\PY{l+m+mi}{4}\PY{p}{,}  \PY{n}{tiles}\PY{o}{=}\PY{l+s+s1}{\PYZsq{}}\PY{l+s+s1}{Stamen Terrain}\PY{l+s+s1}{\PYZsq{}}\PY{p}{)}
\PY{n}{mexico\PYZus{}map}
\end{Verbatim}
\end{tcolorbox}

            \begin{tcolorbox}[breakable, size=fbox, boxrule=.5pt, pad at break*=1mm, opacityfill=0]
\prompt{Out}{outcolor}{11}{\boxspacing}
\begin{Verbatim}[commandchars=\\\{\}]
<folium.folium.Map at 0x7faf19feff60>
\end{Verbatim}
\end{tcolorbox}
        
    Double-click \textbf{here} for the solution.

    \hypertarget{maps-with-markers}{%
\section{\texorpdfstring{Maps with Markers
}{Maps with Markers }}\label{maps-with-markers}}

    Let's download and import the data on police department incidents using
\emph{pandas} \texttt{read\_csv()} method.

    Download the dataset and read it into a \emph{pandas} dataframe:

    \begin{tcolorbox}[breakable, size=fbox, boxrule=1pt, pad at break*=1mm,colback=cellbackground, colframe=cellborder]
\prompt{In}{incolor}{12}{\boxspacing}
\begin{Verbatim}[commandchars=\\\{\}]
\PY{n}{df\PYZus{}incidents} \PY{o}{=} \PY{n}{pd}\PY{o}{.}\PY{n}{read\PYZus{}csv}\PY{p}{(}\PY{l+s+s1}{\PYZsq{}}\PY{l+s+s1}{https://s3\PYZhy{}api.us\PYZhy{}geo.objectstorage.softlayer.net/cf\PYZhy{}courses\PYZhy{}data/CognitiveClass/DV0101EN/labs/Data\PYZus{}Files/Police\PYZus{}Department\PYZus{}Incidents\PYZus{}\PYZhy{}\PYZus{}Previous\PYZus{}Year\PYZus{}\PYZus{}2016\PYZus{}.csv}\PY{l+s+s1}{\PYZsq{}}\PY{p}{)}

\PY{n+nb}{print}\PY{p}{(}\PY{l+s+s1}{\PYZsq{}}\PY{l+s+s1}{Dataset downloaded and read into a pandas dataframe!}\PY{l+s+s1}{\PYZsq{}}\PY{p}{)}
\end{Verbatim}
\end{tcolorbox}

    \begin{Verbatim}[commandchars=\\\{\}]
Dataset downloaded and read into a pandas dataframe!
    \end{Verbatim}

    Let's take a look at the first five items in our dataset.

    \begin{tcolorbox}[breakable, size=fbox, boxrule=1pt, pad at break*=1mm,colback=cellbackground, colframe=cellborder]
\prompt{In}{incolor}{13}{\boxspacing}
\begin{Verbatim}[commandchars=\\\{\}]
\PY{n}{df\PYZus{}incidents}\PY{o}{.}\PY{n}{head}\PY{p}{(}\PY{p}{)}
\end{Verbatim}
\end{tcolorbox}

            \begin{tcolorbox}[breakable, size=fbox, boxrule=.5pt, pad at break*=1mm, opacityfill=0]
\prompt{Out}{outcolor}{13}{\boxspacing}
\begin{Verbatim}[commandchars=\\\{\}]
   IncidntNum      Category                                        Descript  \textbackslash{}
0   120058272   WEAPON LAWS                       POSS OF PROHIBITED WEAPON
1   120058272   WEAPON LAWS  FIREARM, LOADED, IN VEHICLE, POSSESSION OR USE
2   141059263      WARRANTS                                  WARRANT ARREST
3   160013662  NON-CRIMINAL                                   LOST PROPERTY
4   160002740  NON-CRIMINAL                                   LOST PROPERTY

  DayOfWeek                    Date   Time  PdDistrict      Resolution  \textbackslash{}
0    Friday  01/29/2016 12:00:00 AM  11:00    SOUTHERN  ARREST, BOOKED
1    Friday  01/29/2016 12:00:00 AM  11:00    SOUTHERN  ARREST, BOOKED
2    Monday  04/25/2016 12:00:00 AM  14:59     BAYVIEW  ARREST, BOOKED
3   Tuesday  01/05/2016 12:00:00 AM  23:50  TENDERLOIN            NONE
4    Friday  01/01/2016 12:00:00 AM  00:30     MISSION            NONE

                  Address           X          Y  \textbackslash{}
0  800 Block of BRYANT ST -122.403405  37.775421
1  800 Block of BRYANT ST -122.403405  37.775421
2   KEITH ST / SHAFTER AV -122.388856  37.729981
3  JONES ST / OFARRELL ST -122.412971  37.785788
4    16TH ST / MISSION ST -122.419672  37.765050

                                Location            PdId
0   (37.775420706711, -122.403404791479)  12005827212120
1   (37.775420706711, -122.403404791479)  12005827212168
2  (37.7299809672996, -122.388856204292)  14105926363010
3  (37.7857883766888, -122.412970537591)  16001366271000
4  (37.7650501214668, -122.419671780296)  16000274071000
\end{Verbatim}
\end{tcolorbox}
        
    So each row consists of 13 features: \textgreater{} 1.
\textbf{IncidntNum}: Incident Number \textgreater{} 2.
\textbf{Category}: Category of crime or incident \textgreater{} 3.
\textbf{Descript}: Description of the crime or incident \textgreater{}
4. \textbf{DayOfWeek}: The day of week on which the incident occurred
\textgreater{} 5. \textbf{Date}: The Date on which the incident occurred
\textgreater{} 6. \textbf{Time}: The time of day on which the incident
occurred \textgreater{} 7. \textbf{PdDistrict}: The police department
district \textgreater{} 8. \textbf{Resolution}: The resolution of the
crime in terms whether the perpetrator was arrested or not
\textgreater{} 9. \textbf{Address}: The closest address to where the
incident took place \textgreater{} 10. \textbf{X}: The longitude value
of the crime location \textgreater{} 11. \textbf{Y}: The latitude value
of the crime location \textgreater{} 12. \textbf{Location}: A tuple of
the latitude and the longitude values \textgreater{} 13. \textbf{PdId}:
The police department ID

    Let's find out how many entries there are in our dataset.

    \begin{tcolorbox}[breakable, size=fbox, boxrule=1pt, pad at break*=1mm,colback=cellbackground, colframe=cellborder]
\prompt{In}{incolor}{14}{\boxspacing}
\begin{Verbatim}[commandchars=\\\{\}]
\PY{n}{df\PYZus{}incidents}\PY{o}{.}\PY{n}{shape}
\end{Verbatim}
\end{tcolorbox}

            \begin{tcolorbox}[breakable, size=fbox, boxrule=.5pt, pad at break*=1mm, opacityfill=0]
\prompt{Out}{outcolor}{14}{\boxspacing}
\begin{Verbatim}[commandchars=\\\{\}]
(150500, 13)
\end{Verbatim}
\end{tcolorbox}
        
    So the dataframe consists of 150,500 crimes, which took place in the
year 2016. In order to reduce computational cost, let's just work with
the first 100 incidents in this dataset.

    \begin{tcolorbox}[breakable, size=fbox, boxrule=1pt, pad at break*=1mm,colback=cellbackground, colframe=cellborder]
\prompt{In}{incolor}{15}{\boxspacing}
\begin{Verbatim}[commandchars=\\\{\}]
\PY{c+c1}{\PYZsh{} get the first 100 crimes in the df\PYZus{}incidents dataframe}
\PY{n}{limit} \PY{o}{=} \PY{l+m+mi}{100}
\PY{n}{df\PYZus{}incidents} \PY{o}{=} \PY{n}{df\PYZus{}incidents}\PY{o}{.}\PY{n}{iloc}\PY{p}{[}\PY{l+m+mi}{0}\PY{p}{:}\PY{n}{limit}\PY{p}{,} \PY{p}{:}\PY{p}{]}
\end{Verbatim}
\end{tcolorbox}

    Let's confirm that our dataframe now consists only of 100 crimes.

    \begin{tcolorbox}[breakable, size=fbox, boxrule=1pt, pad at break*=1mm,colback=cellbackground, colframe=cellborder]
\prompt{In}{incolor}{16}{\boxspacing}
\begin{Verbatim}[commandchars=\\\{\}]
\PY{n}{df\PYZus{}incidents}\PY{o}{.}\PY{n}{shape}
\end{Verbatim}
\end{tcolorbox}

            \begin{tcolorbox}[breakable, size=fbox, boxrule=.5pt, pad at break*=1mm, opacityfill=0]
\prompt{Out}{outcolor}{16}{\boxspacing}
\begin{Verbatim}[commandchars=\\\{\}]
(100, 13)
\end{Verbatim}
\end{tcolorbox}
        
    Now that we reduced the data a little bit, let's visualize where these
crimes took place in the city of San Francisco. We will use the default
style and we will initialize the zoom level to 12.

    \begin{tcolorbox}[breakable, size=fbox, boxrule=1pt, pad at break*=1mm,colback=cellbackground, colframe=cellborder]
\prompt{In}{incolor}{17}{\boxspacing}
\begin{Verbatim}[commandchars=\\\{\}]
\PY{c+c1}{\PYZsh{} San Francisco latitude and longitude values}
\PY{n}{latitude} \PY{o}{=} \PY{l+m+mf}{37.77}
\PY{n}{longitude} \PY{o}{=} \PY{o}{\PYZhy{}}\PY{l+m+mf}{122.42}
\end{Verbatim}
\end{tcolorbox}

    \begin{tcolorbox}[breakable, size=fbox, boxrule=1pt, pad at break*=1mm,colback=cellbackground, colframe=cellborder]
\prompt{In}{incolor}{18}{\boxspacing}
\begin{Verbatim}[commandchars=\\\{\}]
\PY{c+c1}{\PYZsh{} create map and display it}
\PY{n}{sanfran\PYZus{}map} \PY{o}{=} \PY{n}{folium}\PY{o}{.}\PY{n}{Map}\PY{p}{(}\PY{n}{location}\PY{o}{=}\PY{p}{[}\PY{n}{latitude}\PY{p}{,} \PY{n}{longitude}\PY{p}{]}\PY{p}{,} \PY{n}{zoom\PYZus{}start}\PY{o}{=}\PY{l+m+mi}{12}\PY{p}{)}

\PY{c+c1}{\PYZsh{} display the map of San Francisco}
\PY{n}{sanfran\PYZus{}map}
\end{Verbatim}
\end{tcolorbox}

            \begin{tcolorbox}[breakable, size=fbox, boxrule=.5pt, pad at break*=1mm, opacityfill=0]
\prompt{Out}{outcolor}{18}{\boxspacing}
\begin{Verbatim}[commandchars=\\\{\}]
<folium.folium.Map at 0x7faf199daeb8>
\end{Verbatim}
\end{tcolorbox}
        
    Now let's superimpose the locations of the crimes onto the map. The way
to do that in \textbf{Folium} is to create a \emph{feature group} with
its own features and style and then add it to the sanfran\_map.

    \begin{tcolorbox}[breakable, size=fbox, boxrule=1pt, pad at break*=1mm,colback=cellbackground, colframe=cellborder]
\prompt{In}{incolor}{19}{\boxspacing}
\begin{Verbatim}[commandchars=\\\{\}]
\PY{c+c1}{\PYZsh{} instantiate a feature group for the incidents in the dataframe}
\PY{n}{incidents} \PY{o}{=} \PY{n}{folium}\PY{o}{.}\PY{n}{map}\PY{o}{.}\PY{n}{FeatureGroup}\PY{p}{(}\PY{p}{)}

\PY{c+c1}{\PYZsh{} loop through the 100 crimes and add each to the incidents feature group}
\PY{k}{for} \PY{n}{lat}\PY{p}{,} \PY{n}{lng}\PY{p}{,} \PY{o+ow}{in} \PY{n+nb}{zip}\PY{p}{(}\PY{n}{df\PYZus{}incidents}\PY{o}{.}\PY{n}{Y}\PY{p}{,} \PY{n}{df\PYZus{}incidents}\PY{o}{.}\PY{n}{X}\PY{p}{)}\PY{p}{:}
    \PY{n}{incidents}\PY{o}{.}\PY{n}{add\PYZus{}child}\PY{p}{(}
        \PY{n}{folium}\PY{o}{.}\PY{n}{features}\PY{o}{.}\PY{n}{CircleMarker}\PY{p}{(}
            \PY{p}{[}\PY{n}{lat}\PY{p}{,} \PY{n}{lng}\PY{p}{]}\PY{p}{,}
            \PY{n}{radius}\PY{o}{=}\PY{l+m+mi}{5}\PY{p}{,} \PY{c+c1}{\PYZsh{} define how big you want the circle markers to be}
            \PY{n}{color}\PY{o}{=}\PY{l+s+s1}{\PYZsq{}}\PY{l+s+s1}{yellow}\PY{l+s+s1}{\PYZsq{}}\PY{p}{,}
            \PY{n}{fill}\PY{o}{=}\PY{k+kc}{True}\PY{p}{,}
            \PY{n}{fill\PYZus{}color}\PY{o}{=}\PY{l+s+s1}{\PYZsq{}}\PY{l+s+s1}{blue}\PY{l+s+s1}{\PYZsq{}}\PY{p}{,}
            \PY{n}{fill\PYZus{}opacity}\PY{o}{=}\PY{l+m+mf}{0.6}
        \PY{p}{)}
    \PY{p}{)}

\PY{c+c1}{\PYZsh{} add incidents to map}
\PY{n}{sanfran\PYZus{}map}\PY{o}{.}\PY{n}{add\PYZus{}child}\PY{p}{(}\PY{n}{incidents}\PY{p}{)}
\end{Verbatim}
\end{tcolorbox}

            \begin{tcolorbox}[breakable, size=fbox, boxrule=.5pt, pad at break*=1mm, opacityfill=0]
\prompt{Out}{outcolor}{19}{\boxspacing}
\begin{Verbatim}[commandchars=\\\{\}]
<folium.folium.Map at 0x7faf199daeb8>
\end{Verbatim}
\end{tcolorbox}
        
    You can also add some pop-up text that would get displayed when you
hover over a marker. Let's make each marker display the category of the
crime when hovered over.

    \begin{tcolorbox}[breakable, size=fbox, boxrule=1pt, pad at break*=1mm,colback=cellbackground, colframe=cellborder]
\prompt{In}{incolor}{20}{\boxspacing}
\begin{Verbatim}[commandchars=\\\{\}]
\PY{c+c1}{\PYZsh{} instantiate a feature group for the incidents in the dataframe}
\PY{n}{incidents} \PY{o}{=} \PY{n}{folium}\PY{o}{.}\PY{n}{map}\PY{o}{.}\PY{n}{FeatureGroup}\PY{p}{(}\PY{p}{)}

\PY{c+c1}{\PYZsh{} loop through the 100 crimes and add each to the incidents feature group}
\PY{k}{for} \PY{n}{lat}\PY{p}{,} \PY{n}{lng}\PY{p}{,} \PY{o+ow}{in} \PY{n+nb}{zip}\PY{p}{(}\PY{n}{df\PYZus{}incidents}\PY{o}{.}\PY{n}{Y}\PY{p}{,} \PY{n}{df\PYZus{}incidents}\PY{o}{.}\PY{n}{X}\PY{p}{)}\PY{p}{:}
    \PY{n}{incidents}\PY{o}{.}\PY{n}{add\PYZus{}child}\PY{p}{(}
        \PY{n}{folium}\PY{o}{.}\PY{n}{features}\PY{o}{.}\PY{n}{CircleMarker}\PY{p}{(}
            \PY{p}{[}\PY{n}{lat}\PY{p}{,} \PY{n}{lng}\PY{p}{]}\PY{p}{,}
            \PY{n}{radius}\PY{o}{=}\PY{l+m+mi}{5}\PY{p}{,} \PY{c+c1}{\PYZsh{} define how big you want the circle markers to be}
            \PY{n}{color}\PY{o}{=}\PY{l+s+s1}{\PYZsq{}}\PY{l+s+s1}{yellow}\PY{l+s+s1}{\PYZsq{}}\PY{p}{,}
            \PY{n}{fill}\PY{o}{=}\PY{k+kc}{True}\PY{p}{,}
            \PY{n}{fill\PYZus{}color}\PY{o}{=}\PY{l+s+s1}{\PYZsq{}}\PY{l+s+s1}{blue}\PY{l+s+s1}{\PYZsq{}}\PY{p}{,}
            \PY{n}{fill\PYZus{}opacity}\PY{o}{=}\PY{l+m+mf}{0.6}
        \PY{p}{)}
    \PY{p}{)}

\PY{c+c1}{\PYZsh{} add pop\PYZhy{}up text to each marker on the map}
\PY{n}{latitudes} \PY{o}{=} \PY{n+nb}{list}\PY{p}{(}\PY{n}{df\PYZus{}incidents}\PY{o}{.}\PY{n}{Y}\PY{p}{)}
\PY{n}{longitudes} \PY{o}{=} \PY{n+nb}{list}\PY{p}{(}\PY{n}{df\PYZus{}incidents}\PY{o}{.}\PY{n}{X}\PY{p}{)}
\PY{n}{labels} \PY{o}{=} \PY{n+nb}{list}\PY{p}{(}\PY{n}{df\PYZus{}incidents}\PY{o}{.}\PY{n}{Category}\PY{p}{)}

\PY{k}{for} \PY{n}{lat}\PY{p}{,} \PY{n}{lng}\PY{p}{,} \PY{n}{label} \PY{o+ow}{in} \PY{n+nb}{zip}\PY{p}{(}\PY{n}{latitudes}\PY{p}{,} \PY{n}{longitudes}\PY{p}{,} \PY{n}{labels}\PY{p}{)}\PY{p}{:}
    \PY{n}{folium}\PY{o}{.}\PY{n}{Marker}\PY{p}{(}\PY{p}{[}\PY{n}{lat}\PY{p}{,} \PY{n}{lng}\PY{p}{]}\PY{p}{,} \PY{n}{popup}\PY{o}{=}\PY{n}{label}\PY{p}{)}\PY{o}{.}\PY{n}{add\PYZus{}to}\PY{p}{(}\PY{n}{sanfran\PYZus{}map}\PY{p}{)}    
    
\PY{c+c1}{\PYZsh{} add incidents to map}
\PY{n}{sanfran\PYZus{}map}\PY{o}{.}\PY{n}{add\PYZus{}child}\PY{p}{(}\PY{n}{incidents}\PY{p}{)}
\end{Verbatim}
\end{tcolorbox}

            \begin{tcolorbox}[breakable, size=fbox, boxrule=.5pt, pad at break*=1mm, opacityfill=0]
\prompt{Out}{outcolor}{20}{\boxspacing}
\begin{Verbatim}[commandchars=\\\{\}]
<folium.folium.Map at 0x7faf199daeb8>
\end{Verbatim}
\end{tcolorbox}
        
    Isn't this really cool? Now you are able to know what crime category
occurred at each marker.

If you find the map to be so congested will all these markers, there are
two remedies to this problem. The simpler solution is to remove these
location markers and just add the text to the circle markers themselves
as follows:

    \begin{tcolorbox}[breakable, size=fbox, boxrule=1pt, pad at break*=1mm,colback=cellbackground, colframe=cellborder]
\prompt{In}{incolor}{21}{\boxspacing}
\begin{Verbatim}[commandchars=\\\{\}]
\PY{c+c1}{\PYZsh{} create map and display it}
\PY{n}{sanfran\PYZus{}map} \PY{o}{=} \PY{n}{folium}\PY{o}{.}\PY{n}{Map}\PY{p}{(}\PY{n}{location}\PY{o}{=}\PY{p}{[}\PY{n}{latitude}\PY{p}{,} \PY{n}{longitude}\PY{p}{]}\PY{p}{,} \PY{n}{zoom\PYZus{}start}\PY{o}{=}\PY{l+m+mi}{12}\PY{p}{)}

\PY{c+c1}{\PYZsh{} loop through the 100 crimes and add each to the map}
\PY{k}{for} \PY{n}{lat}\PY{p}{,} \PY{n}{lng}\PY{p}{,} \PY{n}{label} \PY{o+ow}{in} \PY{n+nb}{zip}\PY{p}{(}\PY{n}{df\PYZus{}incidents}\PY{o}{.}\PY{n}{Y}\PY{p}{,} \PY{n}{df\PYZus{}incidents}\PY{o}{.}\PY{n}{X}\PY{p}{,} \PY{n}{df\PYZus{}incidents}\PY{o}{.}\PY{n}{Category}\PY{p}{)}\PY{p}{:}
    \PY{n}{folium}\PY{o}{.}\PY{n}{features}\PY{o}{.}\PY{n}{CircleMarker}\PY{p}{(}
        \PY{p}{[}\PY{n}{lat}\PY{p}{,} \PY{n}{lng}\PY{p}{]}\PY{p}{,}
        \PY{n}{radius}\PY{o}{=}\PY{l+m+mi}{5}\PY{p}{,} \PY{c+c1}{\PYZsh{} define how big you want the circle markers to be}
        \PY{n}{color}\PY{o}{=}\PY{l+s+s1}{\PYZsq{}}\PY{l+s+s1}{yellow}\PY{l+s+s1}{\PYZsq{}}\PY{p}{,}
        \PY{n}{fill}\PY{o}{=}\PY{k+kc}{True}\PY{p}{,}
        \PY{n}{popup}\PY{o}{=}\PY{n}{label}\PY{p}{,}
        \PY{n}{fill\PYZus{}color}\PY{o}{=}\PY{l+s+s1}{\PYZsq{}}\PY{l+s+s1}{blue}\PY{l+s+s1}{\PYZsq{}}\PY{p}{,}
        \PY{n}{fill\PYZus{}opacity}\PY{o}{=}\PY{l+m+mf}{0.6}
    \PY{p}{)}\PY{o}{.}\PY{n}{add\PYZus{}to}\PY{p}{(}\PY{n}{sanfran\PYZus{}map}\PY{p}{)}

\PY{c+c1}{\PYZsh{} show map}
\PY{n}{sanfran\PYZus{}map}
\end{Verbatim}
\end{tcolorbox}

            \begin{tcolorbox}[breakable, size=fbox, boxrule=.5pt, pad at break*=1mm, opacityfill=0]
\prompt{Out}{outcolor}{21}{\boxspacing}
\begin{Verbatim}[commandchars=\\\{\}]
<folium.folium.Map at 0x7faf153f4ef0>
\end{Verbatim}
\end{tcolorbox}
        
    The other proper remedy is to group the markers into different clusters.
Each cluster is then represented by the number of crimes in each
neighborhood. These clusters can be thought of as pockets of San
Francisco which you can then analyze separately.

To implement this, we start off by instantiating a \emph{MarkerCluster}
object and adding all the data points in the dataframe to this object.

    \begin{tcolorbox}[breakable, size=fbox, boxrule=1pt, pad at break*=1mm,colback=cellbackground, colframe=cellborder]
\prompt{In}{incolor}{22}{\boxspacing}
\begin{Verbatim}[commandchars=\\\{\}]
\PY{k+kn}{from} \PY{n+nn}{folium} \PY{k+kn}{import} \PY{n}{plugins}

\PY{c+c1}{\PYZsh{} let\PYZsq{}s start again with a clean copy of the map of San Francisco}
\PY{n}{sanfran\PYZus{}map} \PY{o}{=} \PY{n}{folium}\PY{o}{.}\PY{n}{Map}\PY{p}{(}\PY{n}{location} \PY{o}{=} \PY{p}{[}\PY{n}{latitude}\PY{p}{,} \PY{n}{longitude}\PY{p}{]}\PY{p}{,} \PY{n}{zoom\PYZus{}start} \PY{o}{=} \PY{l+m+mi}{12}\PY{p}{)}

\PY{c+c1}{\PYZsh{} instantiate a mark cluster object for the incidents in the dataframe}
\PY{n}{incidents} \PY{o}{=} \PY{n}{plugins}\PY{o}{.}\PY{n}{MarkerCluster}\PY{p}{(}\PY{p}{)}\PY{o}{.}\PY{n}{add\PYZus{}to}\PY{p}{(}\PY{n}{sanfran\PYZus{}map}\PY{p}{)}

\PY{c+c1}{\PYZsh{} loop through the dataframe and add each data point to the mark cluster}
\PY{k}{for} \PY{n}{lat}\PY{p}{,} \PY{n}{lng}\PY{p}{,} \PY{n}{label}\PY{p}{,} \PY{o+ow}{in} \PY{n+nb}{zip}\PY{p}{(}\PY{n}{df\PYZus{}incidents}\PY{o}{.}\PY{n}{Y}\PY{p}{,} \PY{n}{df\PYZus{}incidents}\PY{o}{.}\PY{n}{X}\PY{p}{,} \PY{n}{df\PYZus{}incidents}\PY{o}{.}\PY{n}{Category}\PY{p}{)}\PY{p}{:}
    \PY{n}{folium}\PY{o}{.}\PY{n}{Marker}\PY{p}{(}
        \PY{n}{location}\PY{o}{=}\PY{p}{[}\PY{n}{lat}\PY{p}{,} \PY{n}{lng}\PY{p}{]}\PY{p}{,}
        \PY{n}{icon}\PY{o}{=}\PY{k+kc}{None}\PY{p}{,}
        \PY{n}{popup}\PY{o}{=}\PY{n}{label}\PY{p}{,}
    \PY{p}{)}\PY{o}{.}\PY{n}{add\PYZus{}to}\PY{p}{(}\PY{n}{incidents}\PY{p}{)}

\PY{c+c1}{\PYZsh{} display map}
\PY{n}{sanfran\PYZus{}map}
\end{Verbatim}
\end{tcolorbox}

            \begin{tcolorbox}[breakable, size=fbox, boxrule=.5pt, pad at break*=1mm, opacityfill=0]
\prompt{Out}{outcolor}{22}{\boxspacing}
\begin{Verbatim}[commandchars=\\\{\}]
<folium.folium.Map at 0x7faf144f2828>
\end{Verbatim}
\end{tcolorbox}
        
    \begin{tcolorbox}[breakable, size=fbox, boxrule=1pt, pad at break*=1mm,colback=cellbackground, colframe=cellborder]
\prompt{In}{incolor}{24}{\boxspacing}
\begin{Verbatim}[commandchars=\\\{\}]
\PY{o}{!}conda install \PYZhy{}c anaconda xlrd \PYZhy{}\PYZhy{}yes
\end{Verbatim}
\end{tcolorbox}

    \begin{Verbatim}[commandchars=\\\{\}]
Solving environment: done


==> WARNING: A newer version of conda exists. <==
  current version: 4.5.11
  latest version: 4.8.0

Please update conda by running

    \$ conda update -n base -c defaults conda



\#\# Package Plan \#\#

  environment location: /home/jupyterlab/conda/envs/python

  added / updated specs:
    - xlrd


The following packages will be downloaded:

    package                    |            build
    ---------------------------|-----------------
    openssl-1.1.1              |       h7b6447c\_0         5.0 MB  anaconda
    xlrd-1.2.0                 |             py\_0         108 KB  anaconda
    certifi-2019.11.28         |           py36\_0         156 KB  anaconda
    ------------------------------------------------------------
                                           Total:         5.3 MB

The following packages will be UPDATED:

    certifi: 2019.11.28-py36\_0 conda-forge --> 2019.11.28-py36\_0 anaconda
    openssl: 1.1.1d-h516909a\_0 conda-forge --> 1.1.1-h7b6447c\_0  anaconda
    xlrd:    1.1.0-py37\_1                  --> 1.2.0-py\_0        anaconda


Downloading and Extracting Packages
openssl-1.1.1        | 5.0 MB    | \#\#\#\#\#\#\#\#\#\#\#\#\#\#\#\#\#\#\#\#\#\#\#\#\#\#\#\#\#\#\#\#\#\#\#\#\# | 100\%
xlrd-1.2.0           | 108 KB    | \#\#\#\#\#\#\#\#\#\#\#\#\#\#\#\#\#\#\#\#\#\#\#\#\#\#\#\#\#\#\#\#\#\#\#\#\# | 100\%
certifi-2019.11.28   | 156 KB    | \#\#\#\#\#\#\#\#\#\#\#\#\#\#\#\#\#\#\#\#\#\#\#\#\#\#\#\#\#\#\#\#\#\#\#\#\# | 100\%
Preparing transaction: done
Verifying transaction: done
Executing transaction: done
    \end{Verbatim}

    Notice how when you zoom out all the way, all markers are grouped into
one cluster, \emph{the global cluster}, of 100 markers or crimes, which
is the total number of crimes in our dataframe. Once you start zooming
in, the \emph{global cluster} will start breaking up into smaller
clusters. Zooming in all the way will result in individual markers.

    \hypertarget{choropleth-maps}{%
\section{\texorpdfstring{Choropleth Maps
}{Choropleth Maps }}\label{choropleth-maps}}

A \texttt{Choropleth} map is a thematic map in which areas are shaded or
patterned in proportion to the measurement of the statistical variable
being displayed on the map, such as population density or per-capita
income. The choropleth map provides an easy way to visualize how a
measurement varies across a geographic area or it shows the level of
variability within a region. Below is a \texttt{Choropleth} map of the
US depicting the population by square mile per state.

    Now, let's create our own \texttt{Choropleth} map of the world depicting
immigration from various countries to Canada.

Let's first download and import our primary Canadian immigration dataset
using \emph{pandas} \texttt{read\_excel()} method. Normally, before we
can do that, we would need to download a module which \emph{pandas}
requires to read in excel files. This module is \textbf{xlrd}. For your
convenience, we have pre-installed this module, so you would not have to
worry about that. Otherwise, you would need to run the following line of
code to install the \textbf{xlrd} module:

\begin{verbatim}
!conda install -c anaconda xlrd --yes
\end{verbatim}

    Download the dataset and read it into a \emph{pandas} dataframe:

    \begin{tcolorbox}[breakable, size=fbox, boxrule=1pt, pad at break*=1mm,colback=cellbackground, colframe=cellborder]
\prompt{In}{incolor}{25}{\boxspacing}
\begin{Verbatim}[commandchars=\\\{\}]
\PY{n}{df\PYZus{}can} \PY{o}{=} \PY{n}{pd}\PY{o}{.}\PY{n}{read\PYZus{}excel}\PY{p}{(}\PY{l+s+s1}{\PYZsq{}}\PY{l+s+s1}{https://s3\PYZhy{}api.us\PYZhy{}geo.objectstorage.softlayer.net/cf\PYZhy{}courses\PYZhy{}data/CognitiveClass/DV0101EN/labs/Data\PYZus{}Files/Canada.xlsx}\PY{l+s+s1}{\PYZsq{}}\PY{p}{,}
                     \PY{n}{sheet\PYZus{}name}\PY{o}{=}\PY{l+s+s1}{\PYZsq{}}\PY{l+s+s1}{Canada by Citizenship}\PY{l+s+s1}{\PYZsq{}}\PY{p}{,}
                     \PY{n}{skiprows}\PY{o}{=}\PY{n+nb}{range}\PY{p}{(}\PY{l+m+mi}{20}\PY{p}{)}\PY{p}{,}
                     \PY{n}{skipfooter}\PY{o}{=}\PY{l+m+mi}{2}\PY{p}{)}

\PY{n+nb}{print}\PY{p}{(}\PY{l+s+s1}{\PYZsq{}}\PY{l+s+s1}{Data downloaded and read into a dataframe!}\PY{l+s+s1}{\PYZsq{}}\PY{p}{)}
\end{Verbatim}
\end{tcolorbox}

    \begin{Verbatim}[commandchars=\\\{\}]
Data downloaded and read into a dataframe!
    \end{Verbatim}

    Let's take a look at the first five items in our dataset.

    \begin{tcolorbox}[breakable, size=fbox, boxrule=1pt, pad at break*=1mm,colback=cellbackground, colframe=cellborder]
\prompt{In}{incolor}{26}{\boxspacing}
\begin{Verbatim}[commandchars=\\\{\}]
\PY{n}{df\PYZus{}can}\PY{o}{.}\PY{n}{head}\PY{p}{(}\PY{p}{)}
\end{Verbatim}
\end{tcolorbox}

            \begin{tcolorbox}[breakable, size=fbox, boxrule=.5pt, pad at break*=1mm, opacityfill=0]
\prompt{Out}{outcolor}{26}{\boxspacing}
\begin{Verbatim}[commandchars=\\\{\}]
         Type    Coverage          OdName  AREA AreaName   REG  \textbackslash{}
0  Immigrants  Foreigners     Afghanistan   935     Asia  5501
1  Immigrants  Foreigners         Albania   908   Europe   925
2  Immigrants  Foreigners         Algeria   903   Africa   912
3  Immigrants  Foreigners  American Samoa   909  Oceania   957
4  Immigrants  Foreigners         Andorra   908   Europe   925

           RegName  DEV             DevName  1980  {\ldots}  2004  2005  2006  \textbackslash{}
0    Southern Asia  902  Developing regions    16  {\ldots}  2978  3436  3009
1  Southern Europe  901   Developed regions     1  {\ldots}  1450  1223   856
2  Northern Africa  902  Developing regions    80  {\ldots}  3616  3626  4807
3        Polynesia  902  Developing regions     0  {\ldots}     0     0     1
4  Southern Europe  901   Developed regions     0  {\ldots}     0     0     1

   2007  2008  2009  2010  2011  2012  2013
0  2652  2111  1746  1758  2203  2635  2004
1   702   560   716   561   539   620   603
2  3623  4005  5393  4752  4325  3774  4331
3     0     0     0     0     0     0     0
4     1     0     0     0     0     1     1

[5 rows x 43 columns]
\end{Verbatim}
\end{tcolorbox}
        
    Let's find out how many entries there are in our dataset.

    \begin{tcolorbox}[breakable, size=fbox, boxrule=1pt, pad at break*=1mm,colback=cellbackground, colframe=cellborder]
\prompt{In}{incolor}{27}{\boxspacing}
\begin{Verbatim}[commandchars=\\\{\}]
\PY{c+c1}{\PYZsh{} print the dimensions of the dataframe}
\PY{n+nb}{print}\PY{p}{(}\PY{n}{df\PYZus{}can}\PY{o}{.}\PY{n}{shape}\PY{p}{)}
\end{Verbatim}
\end{tcolorbox}

    \begin{Verbatim}[commandchars=\\\{\}]
(195, 43)
    \end{Verbatim}

    Clean up data. We will make some modifications to the original dataset
to make it easier to create our visualizations. Refer to
\emph{Introduction to Matplotlib and Line Plots} and \emph{Area Plots,
Histograms, and Bar Plots} notebooks for a detailed description of this
preprocessing.

    \begin{tcolorbox}[breakable, size=fbox, boxrule=1pt, pad at break*=1mm,colback=cellbackground, colframe=cellborder]
\prompt{In}{incolor}{28}{\boxspacing}
\begin{Verbatim}[commandchars=\\\{\}]
\PY{c+c1}{\PYZsh{} clean up the dataset to remove unnecessary columns (eg. REG) }
\PY{n}{df\PYZus{}can}\PY{o}{.}\PY{n}{drop}\PY{p}{(}\PY{p}{[}\PY{l+s+s1}{\PYZsq{}}\PY{l+s+s1}{AREA}\PY{l+s+s1}{\PYZsq{}}\PY{p}{,}\PY{l+s+s1}{\PYZsq{}}\PY{l+s+s1}{REG}\PY{l+s+s1}{\PYZsq{}}\PY{p}{,}\PY{l+s+s1}{\PYZsq{}}\PY{l+s+s1}{DEV}\PY{l+s+s1}{\PYZsq{}}\PY{p}{,}\PY{l+s+s1}{\PYZsq{}}\PY{l+s+s1}{Type}\PY{l+s+s1}{\PYZsq{}}\PY{p}{,}\PY{l+s+s1}{\PYZsq{}}\PY{l+s+s1}{Coverage}\PY{l+s+s1}{\PYZsq{}}\PY{p}{]}\PY{p}{,} \PY{n}{axis}\PY{o}{=}\PY{l+m+mi}{1}\PY{p}{,} \PY{n}{inplace}\PY{o}{=}\PY{k+kc}{True}\PY{p}{)}

\PY{c+c1}{\PYZsh{} let\PYZsq{}s rename the columns so that they make sense}
\PY{n}{df\PYZus{}can}\PY{o}{.}\PY{n}{rename}\PY{p}{(}\PY{n}{columns}\PY{o}{=}\PY{p}{\PYZob{}}\PY{l+s+s1}{\PYZsq{}}\PY{l+s+s1}{OdName}\PY{l+s+s1}{\PYZsq{}}\PY{p}{:}\PY{l+s+s1}{\PYZsq{}}\PY{l+s+s1}{Country}\PY{l+s+s1}{\PYZsq{}}\PY{p}{,} \PY{l+s+s1}{\PYZsq{}}\PY{l+s+s1}{AreaName}\PY{l+s+s1}{\PYZsq{}}\PY{p}{:}\PY{l+s+s1}{\PYZsq{}}\PY{l+s+s1}{Continent}\PY{l+s+s1}{\PYZsq{}}\PY{p}{,}\PY{l+s+s1}{\PYZsq{}}\PY{l+s+s1}{RegName}\PY{l+s+s1}{\PYZsq{}}\PY{p}{:}\PY{l+s+s1}{\PYZsq{}}\PY{l+s+s1}{Region}\PY{l+s+s1}{\PYZsq{}}\PY{p}{\PYZcb{}}\PY{p}{,} \PY{n}{inplace}\PY{o}{=}\PY{k+kc}{True}\PY{p}{)}

\PY{c+c1}{\PYZsh{} for sake of consistency, let\PYZsq{}s also make all column labels of type string}
\PY{n}{df\PYZus{}can}\PY{o}{.}\PY{n}{columns} \PY{o}{=} \PY{n+nb}{list}\PY{p}{(}\PY{n+nb}{map}\PY{p}{(}\PY{n+nb}{str}\PY{p}{,} \PY{n}{df\PYZus{}can}\PY{o}{.}\PY{n}{columns}\PY{p}{)}\PY{p}{)}

\PY{c+c1}{\PYZsh{} add total column}
\PY{n}{df\PYZus{}can}\PY{p}{[}\PY{l+s+s1}{\PYZsq{}}\PY{l+s+s1}{Total}\PY{l+s+s1}{\PYZsq{}}\PY{p}{]} \PY{o}{=} \PY{n}{df\PYZus{}can}\PY{o}{.}\PY{n}{sum}\PY{p}{(}\PY{n}{axis}\PY{o}{=}\PY{l+m+mi}{1}\PY{p}{)}

\PY{c+c1}{\PYZsh{} years that we will be using in this lesson \PYZhy{} useful for plotting later on}
\PY{n}{years} \PY{o}{=} \PY{n+nb}{list}\PY{p}{(}\PY{n+nb}{map}\PY{p}{(}\PY{n+nb}{str}\PY{p}{,} \PY{n+nb}{range}\PY{p}{(}\PY{l+m+mi}{1980}\PY{p}{,} \PY{l+m+mi}{2014}\PY{p}{)}\PY{p}{)}\PY{p}{)}
\PY{n+nb}{print} \PY{p}{(}\PY{l+s+s1}{\PYZsq{}}\PY{l+s+s1}{data dimensions:}\PY{l+s+s1}{\PYZsq{}}\PY{p}{,} \PY{n}{df\PYZus{}can}\PY{o}{.}\PY{n}{shape}\PY{p}{)}
\end{Verbatim}
\end{tcolorbox}

    \begin{Verbatim}[commandchars=\\\{\}]
data dimensions: (195, 39)
    \end{Verbatim}

    Let's take a look at the first five items of our cleaned dataframe.

    \begin{tcolorbox}[breakable, size=fbox, boxrule=1pt, pad at break*=1mm,colback=cellbackground, colframe=cellborder]
\prompt{In}{incolor}{29}{\boxspacing}
\begin{Verbatim}[commandchars=\\\{\}]
\PY{n}{df\PYZus{}can}\PY{o}{.}\PY{n}{head}\PY{p}{(}\PY{p}{)}
\end{Verbatim}
\end{tcolorbox}

            \begin{tcolorbox}[breakable, size=fbox, boxrule=.5pt, pad at break*=1mm, opacityfill=0]
\prompt{Out}{outcolor}{29}{\boxspacing}
\begin{Verbatim}[commandchars=\\\{\}]
          Country Continent           Region             DevName  1980  1981  \textbackslash{}
0     Afghanistan      Asia    Southern Asia  Developing regions    16    39
1         Albania    Europe  Southern Europe   Developed regions     1     0
2         Algeria    Africa  Northern Africa  Developing regions    80    67
3  American Samoa   Oceania        Polynesia  Developing regions     0     1
4         Andorra    Europe  Southern Europe   Developed regions     0     0

   1982  1983  1984  1985  {\ldots}  2005  2006  2007  2008  2009  2010  2011  \textbackslash{}
0    39    47    71   340  {\ldots}  3436  3009  2652  2111  1746  1758  2203
1     0     0     0     0  {\ldots}  1223   856   702   560   716   561   539
2    71    69    63    44  {\ldots}  3626  4807  3623  4005  5393  4752  4325
3     0     0     0     0  {\ldots}     0     1     0     0     0     0     0
4     0     0     0     0  {\ldots}     0     1     1     0     0     0     0

   2012  2013  Total
0  2635  2004  58639
1   620   603  15699
2  3774  4331  69439
3     0     0      6
4     1     1     15

[5 rows x 39 columns]
\end{Verbatim}
\end{tcolorbox}
        
    In order to create a \texttt{Choropleth} map, we need a GeoJSON file
that defines the areas/boundaries of the state, county, or country that
we are interested in. In our case, since we are endeavoring to create a
world map, we want a GeoJSON that defines the boundaries of all world
countries. For your convenience, we will be providing you with this
file, so let's go ahead and download it. Let's name it
\textbf{world\_countries.json}.

    \begin{tcolorbox}[breakable, size=fbox, boxrule=1pt, pad at break*=1mm,colback=cellbackground, colframe=cellborder]
\prompt{In}{incolor}{30}{\boxspacing}
\begin{Verbatim}[commandchars=\\\{\}]
\PY{c+c1}{\PYZsh{} download countries geojson file}
\PY{o}{!}wget \PYZhy{}\PYZhy{}quiet https://s3\PYZhy{}api.us\PYZhy{}geo.objectstorage.softlayer.net/cf\PYZhy{}courses\PYZhy{}data/CognitiveClass/DV0101EN/labs/Data\PYZus{}Files/world\PYZus{}countries.json \PYZhy{}O world\PYZus{}countries.json
    
\PY{n+nb}{print}\PY{p}{(}\PY{l+s+s1}{\PYZsq{}}\PY{l+s+s1}{GeoJSON file downloaded!}\PY{l+s+s1}{\PYZsq{}}\PY{p}{)}
\end{Verbatim}
\end{tcolorbox}

    \begin{Verbatim}[commandchars=\\\{\}]
GeoJSON file downloaded!
    \end{Verbatim}

    Now that we have the GeoJSON file, let's create a world map, centered
around \textbf{{[}0, 0{]}} \emph{latitude} and \emph{longitude} values,
with an intial zoom level of 2, and using \emph{Mapbox Bright} style.

    \begin{tcolorbox}[breakable, size=fbox, boxrule=1pt, pad at break*=1mm,colback=cellbackground, colframe=cellborder]
\prompt{In}{incolor}{31}{\boxspacing}
\begin{Verbatim}[commandchars=\\\{\}]
\PY{n}{world\PYZus{}geo} \PY{o}{=} \PY{l+s+sa}{r}\PY{l+s+s1}{\PYZsq{}}\PY{l+s+s1}{world\PYZus{}countries.json}\PY{l+s+s1}{\PYZsq{}} \PY{c+c1}{\PYZsh{} geojson file}

\PY{c+c1}{\PYZsh{} create a plain world map}
\PY{n}{world\PYZus{}map} \PY{o}{=} \PY{n}{folium}\PY{o}{.}\PY{n}{Map}\PY{p}{(}\PY{n}{location}\PY{o}{=}\PY{p}{[}\PY{l+m+mi}{0}\PY{p}{,} \PY{l+m+mi}{0}\PY{p}{]}\PY{p}{,} \PY{n}{zoom\PYZus{}start}\PY{o}{=}\PY{l+m+mi}{2}\PY{p}{,} \PY{n}{tiles}\PY{o}{=}\PY{l+s+s1}{\PYZsq{}}\PY{l+s+s1}{Mapbox Bright}\PY{l+s+s1}{\PYZsq{}}\PY{p}{)}
\end{Verbatim}
\end{tcolorbox}

    And now to create a \texttt{Choropleth} map, we will use the
\emph{choropleth} method with the following main parameters:

\begin{enumerate}
\def\labelenumi{\arabic{enumi}.}
\tightlist
\item
  geo\_data, which is the GeoJSON file.
\item
  data, which is the dataframe containing the data.
\item
  columns, which represents the columns in the dataframe that will be
  used to create the \texttt{Choropleth} map.
\item
  key\_on, which is the key or variable in the GeoJSON file that
  contains the name of the variable of interest. To determine that, you
  will need to open the GeoJSON file using any text editor and note the
  name of the key or variable that contains the name of the countries,
  since the countries are our variable of interest. In this case,
  \textbf{name} is the key in the GeoJSON file that contains the name of
  the countries. Note that this key is case\_sensitive, so you need to
  pass exactly as it exists in the GeoJSON file.
\end{enumerate}

    \begin{tcolorbox}[breakable, size=fbox, boxrule=1pt, pad at break*=1mm,colback=cellbackground, colframe=cellborder]
\prompt{In}{incolor}{32}{\boxspacing}
\begin{Verbatim}[commandchars=\\\{\}]
\PY{c+c1}{\PYZsh{} generate choropleth map using the total immigration of each country to Canada from 1980 to 2013}
\PY{n}{world\PYZus{}map}\PY{o}{.}\PY{n}{choropleth}\PY{p}{(}
    \PY{n}{geo\PYZus{}data}\PY{o}{=}\PY{n}{world\PYZus{}geo}\PY{p}{,}
    \PY{n}{data}\PY{o}{=}\PY{n}{df\PYZus{}can}\PY{p}{,}
    \PY{n}{columns}\PY{o}{=}\PY{p}{[}\PY{l+s+s1}{\PYZsq{}}\PY{l+s+s1}{Country}\PY{l+s+s1}{\PYZsq{}}\PY{p}{,} \PY{l+s+s1}{\PYZsq{}}\PY{l+s+s1}{Total}\PY{l+s+s1}{\PYZsq{}}\PY{p}{]}\PY{p}{,}
    \PY{n}{key\PYZus{}on}\PY{o}{=}\PY{l+s+s1}{\PYZsq{}}\PY{l+s+s1}{feature.properties.name}\PY{l+s+s1}{\PYZsq{}}\PY{p}{,}
    \PY{n}{fill\PYZus{}color}\PY{o}{=}\PY{l+s+s1}{\PYZsq{}}\PY{l+s+s1}{YlOrRd}\PY{l+s+s1}{\PYZsq{}}\PY{p}{,} 
    \PY{n}{fill\PYZus{}opacity}\PY{o}{=}\PY{l+m+mf}{0.7}\PY{p}{,} 
    \PY{n}{line\PYZus{}opacity}\PY{o}{=}\PY{l+m+mf}{0.2}\PY{p}{,}
    \PY{n}{legend\PYZus{}name}\PY{o}{=}\PY{l+s+s1}{\PYZsq{}}\PY{l+s+s1}{Immigration to Canada}\PY{l+s+s1}{\PYZsq{}}
\PY{p}{)}

\PY{c+c1}{\PYZsh{} display map}
\PY{n}{world\PYZus{}map}
\end{Verbatim}
\end{tcolorbox}

            \begin{tcolorbox}[breakable, size=fbox, boxrule=.5pt, pad at break*=1mm, opacityfill=0]
\prompt{Out}{outcolor}{32}{\boxspacing}
\begin{Verbatim}[commandchars=\\\{\}]
<folium.folium.Map at 0x7faf1942f898>
\end{Verbatim}
\end{tcolorbox}
        
    As per our \texttt{Choropleth} map legend, the darker the color of a
country and the closer the color to red, the higher the number of
immigrants from that country. Accordingly, the highest immigration over
the course of 33 years (from 1980 to 2013) was from China, India, and
the Philippines, followed by Poland, Pakistan, and interestingly, the
US.

    Notice how the legend is displaying a negative boundary or threshold.
Let's fix that by defining our own thresholds and starting with 0
instead of -6,918!

    \begin{tcolorbox}[breakable, size=fbox, boxrule=1pt, pad at break*=1mm,colback=cellbackground, colframe=cellborder]
\prompt{In}{incolor}{33}{\boxspacing}
\begin{Verbatim}[commandchars=\\\{\}]
\PY{n}{world\PYZus{}geo} \PY{o}{=} \PY{l+s+sa}{r}\PY{l+s+s1}{\PYZsq{}}\PY{l+s+s1}{world\PYZus{}countries.json}\PY{l+s+s1}{\PYZsq{}}

\PY{c+c1}{\PYZsh{} create a numpy array of length 6 and has linear spacing from the minium total immigration to the maximum total immigration}
\PY{n}{threshold\PYZus{}scale} \PY{o}{=} \PY{n}{np}\PY{o}{.}\PY{n}{linspace}\PY{p}{(}\PY{n}{df\PYZus{}can}\PY{p}{[}\PY{l+s+s1}{\PYZsq{}}\PY{l+s+s1}{Total}\PY{l+s+s1}{\PYZsq{}}\PY{p}{]}\PY{o}{.}\PY{n}{min}\PY{p}{(}\PY{p}{)}\PY{p}{,}
                              \PY{n}{df\PYZus{}can}\PY{p}{[}\PY{l+s+s1}{\PYZsq{}}\PY{l+s+s1}{Total}\PY{l+s+s1}{\PYZsq{}}\PY{p}{]}\PY{o}{.}\PY{n}{max}\PY{p}{(}\PY{p}{)}\PY{p}{,}
                              \PY{l+m+mi}{6}\PY{p}{,} \PY{n}{dtype}\PY{o}{=}\PY{n+nb}{int}\PY{p}{)}
\PY{n}{threshold\PYZus{}scale} \PY{o}{=} \PY{n}{threshold\PYZus{}scale}\PY{o}{.}\PY{n}{tolist}\PY{p}{(}\PY{p}{)} \PY{c+c1}{\PYZsh{} change the numpy array to a list}
\PY{n}{threshold\PYZus{}scale}\PY{p}{[}\PY{o}{\PYZhy{}}\PY{l+m+mi}{1}\PY{p}{]} \PY{o}{=} \PY{n}{threshold\PYZus{}scale}\PY{p}{[}\PY{o}{\PYZhy{}}\PY{l+m+mi}{1}\PY{p}{]} \PY{o}{+} \PY{l+m+mi}{1} \PY{c+c1}{\PYZsh{} make sure that the last value of the list is greater than the maximum immigration}

\PY{c+c1}{\PYZsh{} let Folium determine the scale.}
\PY{n}{world\PYZus{}map} \PY{o}{=} \PY{n}{folium}\PY{o}{.}\PY{n}{Map}\PY{p}{(}\PY{n}{location}\PY{o}{=}\PY{p}{[}\PY{l+m+mi}{0}\PY{p}{,} \PY{l+m+mi}{0}\PY{p}{]}\PY{p}{,} \PY{n}{zoom\PYZus{}start}\PY{o}{=}\PY{l+m+mi}{2}\PY{p}{,} \PY{n}{tiles}\PY{o}{=}\PY{l+s+s1}{\PYZsq{}}\PY{l+s+s1}{Mapbox Bright}\PY{l+s+s1}{\PYZsq{}}\PY{p}{)}
\PY{n}{world\PYZus{}map}\PY{o}{.}\PY{n}{choropleth}\PY{p}{(}
    \PY{n}{geo\PYZus{}data}\PY{o}{=}\PY{n}{world\PYZus{}geo}\PY{p}{,}
    \PY{n}{data}\PY{o}{=}\PY{n}{df\PYZus{}can}\PY{p}{,}
    \PY{n}{columns}\PY{o}{=}\PY{p}{[}\PY{l+s+s1}{\PYZsq{}}\PY{l+s+s1}{Country}\PY{l+s+s1}{\PYZsq{}}\PY{p}{,} \PY{l+s+s1}{\PYZsq{}}\PY{l+s+s1}{Total}\PY{l+s+s1}{\PYZsq{}}\PY{p}{]}\PY{p}{,}
    \PY{n}{key\PYZus{}on}\PY{o}{=}\PY{l+s+s1}{\PYZsq{}}\PY{l+s+s1}{feature.properties.name}\PY{l+s+s1}{\PYZsq{}}\PY{p}{,}
    \PY{n}{threshold\PYZus{}scale}\PY{o}{=}\PY{n}{threshold\PYZus{}scale}\PY{p}{,}
    \PY{n}{fill\PYZus{}color}\PY{o}{=}\PY{l+s+s1}{\PYZsq{}}\PY{l+s+s1}{YlOrRd}\PY{l+s+s1}{\PYZsq{}}\PY{p}{,} 
    \PY{n}{fill\PYZus{}opacity}\PY{o}{=}\PY{l+m+mf}{0.7}\PY{p}{,} 
    \PY{n}{line\PYZus{}opacity}\PY{o}{=}\PY{l+m+mf}{0.2}\PY{p}{,}
    \PY{n}{legend\PYZus{}name}\PY{o}{=}\PY{l+s+s1}{\PYZsq{}}\PY{l+s+s1}{Immigration to Canada}\PY{l+s+s1}{\PYZsq{}}\PY{p}{,}
    \PY{n}{reset}\PY{o}{=}\PY{k+kc}{True}
\PY{p}{)}
\PY{n}{world\PYZus{}map}
\end{Verbatim}
\end{tcolorbox}

            \begin{tcolorbox}[breakable, size=fbox, boxrule=.5pt, pad at break*=1mm, opacityfill=0]
\prompt{Out}{outcolor}{33}{\boxspacing}
\begin{Verbatim}[commandchars=\\\{\}]
<folium.folium.Map at 0x7faf13ee9588>
\end{Verbatim}
\end{tcolorbox}
        
    Much better now! Feel free to play around with the data and perhaps
create \texttt{Choropleth} maps for individuals years, or perhaps
decades, and see how they compare with the entire period from 1980 to
2013.

    \hypertarget{thank-you-for-completing-this-lab}{%
\subsubsection{Thank you for completing this
lab!}\label{thank-you-for-completing-this-lab}}

This notebook was created by
\href{https://www.linkedin.com/in/aklson/}{Alex Aklson}. I hope you
found this lab interesting and educational. Feel free to contact me if
you have any questions!

    This notebook is part of a course on \textbf{Coursera} called \emph{Data
Visualization with Python}. If you accessed this notebook outside the
course, you can take this course online by clicking
\href{http://cocl.us/DV0101EN_Coursera_Week3_LAB2}{here}.

    Copyright © 2019
\href{https://cognitiveclass.ai/?utm_source=bducopyrightlink\&utm_medium=dswb\&utm_campaign=bdu}{Cognitive
Class}. This notebook and its source code are released under the terms
of the \href{https://bigdatauniversity.com/mit-license/}{MIT License}.


    % Add a bibliography block to the postdoc
    
    
    
\end{document}
